\pagebreak
\section{Lista 1 - \date{12/1/2026}}

\begin{problem}
    Dados inteiros positivos $a, b$ e $c$, dois a dois primos entre si, demonstre que $2abc - ab - bc -ca$ é o 
    maior número inteiro que não pode expressar-se na forma $xbc + yca + zab$ com $x,y$ e $z$ inteiros não negativos.
\end{problem}

\begin{proof}
    Note que como $(b,c) = 1$, temos que $(ab,ac) = a$ e, portanto por Bachét-Bezout existe solução
    para $z'ab + y'ca = a$ com $z',y'$ inteiros. Por sua vez, como $(a,bc) = 1$, existe solução para 
    $ma + nbc = 1$ com $m,n$ inteiros. Juntando as duas equações, encontramos $mz'ab + my'ca + nbc = 1$ que é solução para 
    a equação $xbc + yca + zab = 1$ e, portanto, temos soluções para $xbc + yca + zab = k$ para qualquer inteiro $k$.

    Vamos mostrar que $2abc - ab - bc -ca$ não pode ser escrito como $xbc + yca + zab$ para $x,y,z \in \N$. Suponha, que conseguimos,
    temos 
    \begin{align*}
        2abc - ab - bc -ca = xbc + yca + zab\\
        2abc = (x+1)bc + (y+1)ca + (z+1)ab 
    \end{align*}
    tomando a segunda equação módulo $a$, achamos
    \[
        0 \equiv (x+1)bc \pmod a \Rightarrow x+1 \equiv 0 \pmod a
    \]
    ou seja, $a \mid (x+1)$. Como $x \geq 0$, devemos ter $(x+1) \geq a$. Simetricamente (tomando módulo $b$ e depois $c$), sabemos que $(y+1) \geq b$ e $(z+1) \geq c$.
    Mas já encontramos contradição, uma vez que essas desigualdades implicam
    \[
    (x+1)bc + (y+1)ca +(z+1)ab \geq abc + bca + cab = 3abc > 2abc
    \]

    Agora seja $n > 2abc - bc - ac - ab$, mostraremos que existe solução natural para $n = xbc + yac + zab$. Primeiro, vamos
    caracterizar as soluções inteiras, que existem pela observação anterior. Note que se $(x,y,z)$ e $(x',y',z')$ são soluções,
    então
    \begin{equation}
        \label{lista1:eq:prob1}
        (x-x')bc + (y-y')ac + (z-z')ab = 0
    \end{equation}
    tomando a equação módulo $a$, vemos que $(x-x') \equiv 0 \pmod a$ e portanto $x' = x + ra$ para algum $r \in \Z$. Simetricamente,
    vemos que $y' = y + sb$ e $z' = z + tc$ para $s,t \in \Z$. Portanto, [\ref{lista1:eq:prob1}] se expressa como 
    \[
        (ra)bc + (sb)ac + (tc)ab = (r+s+t)abc = 0 \iff (r+s+t) = 0 
    \]
    Ou seja, se $(x_0, y_0, z_0)$ é uma solução inicial, todas as outras soluções são da forma $(x_0 + ra, y_0 + sb, z_0 + tc)$ onde 
    $r + s + t = 0$, é fácil ver que qualquer qualquer tripla dessa forma também satisfaz a equação original. Nosso problema se resume então
    a encontrar soluções inteiras $(r,s,t)$ para a seguinte série de relações:
    \begin{align*}
        x_0 + ra > -1\\
        y_0 + sb > -1\\
        z_0 + tc > -1\\
        r + s + t = 0
    \end{align*}
    Isolando as variáveis e escrevendo $t$ como $-(r+s)$, temos
    \begin{align*}
        -\frac{(x_0+1)}{a} < r\\
        -\frac{(y_0+1)}{b} < s\\
        r+s < \frac{(z_0 + 1)}{c}
    \end{align*}
    As duas primeiras desigualdades, implicam que 
    \[
       - \bigg(\frac{(x_0+1)}{a} + \frac{(y_0+1)}{b}\bigg) < r+s < \frac{(z_0 + 1)}{c}
    \]
    Notamos (seguindo a resolução do livro para um problema similar) que 
    \[
        \frac{(z_0 + 1)}{c} - - \bigg(\frac{(x_0+1)}{a} + \frac{(y_0+1)}{b}\bigg) = \frac{(z_0 + 1)}{c} + \frac{(x_0+1)}{a} + \frac{(y_0+1)}{b} =
        \frac{n + bc + ac + ab}{abc} > 2
    \]
    pois $n > 2abc - bc - ac - ab$. Segue que o intervalo $\bigg(- \frac{(x_0+1)}{a} - \frac{(y_0+1)}{b}, \frac{(z_0 + 1)}{c}\bigg)$ tem ao menos dois inteiros. Tomando
    \begin{align*}
        r = \begin{cases}
            \ceil{-(x_0+1)/a} & \text{se } -(x_0+1)/a \not\in \Z\\
            \ceil{-(x_0+1)/a} + 1 & \text{se } -(x_0+1)/a \in \Z
        \end{cases}
    \end{align*}
    e $s$ análoga, sendo 
    \begin{align*}
        s = \begin{cases}
            \ceil{-(y_0+1)/b} & \text{se } -(y_0+1)/b \not\in \Z\\
            \ceil{-(y_0+1)/b} + 1 & \text{se } -(y_0+1)/b \in \Z
        \end{cases}
    \end{align*}
    achamos soluções $(r,s,t)$ compatíveis com o sistema de desigualdades.

\end{proof}

\begin{problem}
    Seja $p$ um número primo ímpar. Seja $s$ o menor inteiro positivo que não é resíduo quadrático módulo $p$.
    \begin{enumerate}[label=(\alph*)]
        \item Mostre que $p > s^2 - s$.
        \item Suponha que $p > 5$ e que $-1$ seja resíduo quadrático módulo $p$: mostre que $p>2s^2 - s$.
    \end{enumerate}
\end{problem}

\begin{proof}
\textbf{(a)} Como 1 é sempre resíduo quadrático, sabemos que $s \geq 2$. Notamos que, pela propriedade multiplicativa dos símbolos de Legendre,
para todo $1 \leq k \leq (s-1)$, vale que 
\[
    \bigg(\frac{ks}{p}\bigg) = \bigg(\frac{k}{p}\bigg)\bigg(\frac{s}{p}\bigg) = 1\cdot-1 = -1.
\]
Isto é, nenhum dos números $\{s, 2s, \dots (s-1)s\}$ são resíduos quadráticos. Como $p$ é um primo ímpar, temos que ao menos $(p-1)/2$ elementos de $\Z_p$ 
não são resíduos quadráticos, logo $p > s$. Suponha que $p < s(s-1)$, então existe $1 \leq k < (s-1)$ tal que
\[
    sk < p < s(k+1).
\]
Isso é, $s(k+1) = p + r$ onde $0 < r < s$ e temos $s(k+1) \equiv r \pmod p$. Portanto $-1 = \big(\frac{s(k+1)}{p}\big) = \big(\frac{r}{p}\big) = 1$, absurdo.
\end{proof}

\begin{proof}
    \textbf{(b)} Segue muito similarmente da letra anterior. Note que como $p > 5$, se $s = 2$, $p > 2\cdot2^2 - 2 = 6$, já que o próximo primo ímpar é 7.
    Podemos supor que $\big(\frac{2}{p}\big) = 1$ e $s > 2$. Já que temos $\big(\frac{-1}{p}\big) = 1$, sabemos que para todo $1 \leq k \leq (s-1)$,
    \[
        \bigg(\frac{-2sk}{p}\bigg) = \bigg(\frac{2sk}{p}\bigg) = -1.
    \]
    Agora suponha que $p < 2s^2 - s$ ou, posto de forma mais instrutiva, $p < 2s(s-1) + s$. Então existe $1 \leq k \leq (s-1)$ tal que 
    \[
        p \in (2sk - s, 2sk + s).
    \]
    Note que como $p$ é um primo maior que $s$, ele não pode estar nas bordas destes intervalos (que são múltiplas de s).
    Se vale que $2sk - s < p < 2sk$, então $2sk = p + r$ onde $0 < r < s$, logo $-1 = \big(\frac{2ks}{p}\big) = \big(\frac{r}{p}\big) = 1$, o que é absurdo.
    Se por outro lado, vale que $2sk < p < 2sk + s$, então podemos escrever $2sk = p - r$ onde $0 < r < s$ e $2sk \equiv -r \pmod p$, teríamos 
    $-1 = \big(\frac{2ks}{p}\big) = \big(\frac{-r}{p}\big) = 1$, absurdo também.
\end{proof}

A seguinte definição será útil para os próximos dois problemas.
\begin{definition}
    Dado $p$ primo e $n \neq 0$ inteiro, 
    \[
        \nu_p(n) = \max \{ \alpha \in \N: p^\alpha \mid n\}
    \]
\end{definition}

\begin{problem}
    Seja $p$ um primo ímpar, $a$ um inteiro e $n$ um inteiro positivo. Sejam $\alpha$ e $\beta$ inteiros negativos, com $\alpha > 0$. Prove:
    \begin{enumerate}[label=(\alph*)]
        \label{prob:1.3.a}
        \item  Se $p^\beta$ e $p^\alpha$ são as maiores potências de $p$ que dividem $n$ e $a-1$ respectivamente então $p^{\alpha + \beta}$ é 
        a maior potência que divide $a^n - 1$.
        \item Se $n$ é ímpar e $p^\beta$ e $p^\alpha$ são as maiores potências de $p$ que dividem $n$ e $a+1$ respectivamente então $p^{\alpha + \beta}$ é a maior potência
        de $p$ que divide $a^n + 1$.
    \end{enumerate}
\end{problem}

\begin{proof}
    \textbf{(a)} Considere o caso particular $\nu_p(a - 1) = \alpha > 0$ e $\nu_p(n) = \beta = 0$,
    queremos mostrar que $\nu_p(a^n - 1) = \alpha$, temos
    \begin{align*}
        a^n - 1 &= (a-1) \cdot \bigg(\sum_{j=0}^{n-1} a^j \bigg)\\
        \nu_p(a^n - 1) &= \nu_p(a-1) + \nu_p\bigg( \sum_{j=0}^{n-1} a^j \bigg) = \alpha + \nu_p\bigg( \sum_{j=0}^{n-1} a^j \bigg)
    \end{align*}
    então basta mostrar que $\nu_p\big( \sum_{j=0}^{n-1} a^j \big) = 0$. Verificamos que, como $\nu_p(a-1) > 0$, $p \mid (a-1)$, ou seja $a \equiv 1 \pmod p$. Mas então 
    \[
    \sum_{j=0}^{n-1} a^j \equiv \sum_{j=0}^{n-1} 1 \equiv n \not \equiv 0 \pmod p,
    \]
    ou seja $p \nmid \sum_{j=0}^{n-1} a^j$ e $\nu_p\big( \sum_{j=0}^{n-1} a^j \big) = 0$.

    Vamos provar indutivamente para $n = p^\beta$, $\beta \geq 1$, o caso base principal é $n = p$. Queremos mostrar que $\nu_p(a^p - 1) = \nu_p(a-1) + 1$, ou seja, como antes, que $\nu_p\big( \sum_{j=0}^{p-1} a^j \big) = 1$.
    Como $\nu_p(a-1) = \alpha$, escrevemos $a = p^\alpha s + 1$ com $(p,s) = 1$. O somatório se traduz como $\big( \sum_{j=0}^{p-1} (p^\alpha s + 1)^j \big)$. Se $\alpha \geq 2$,
    segue que 
    \[
        \sum_{j=0}^{p-1} (p^\alpha s + 1)^j \equiv \sum_{j=0}^{p-1} 1 \equiv p \pmod {p^2}
    \]
    logo $p \mid \sum_{j=0}^{p-1} (p^\alpha s + 1)^j$, mas $p^2$ não, e portanto $\nu_p\big( \sum_{j=0}^{p-1} a^j \big) = 1$. Se $\alpha = 1$, temos 
    \[
        \sum_{j=0}^{p-1} (ps + 1)^j \equiv \sum_{j=0}^{p-1} (1 + jps) \equiv p + ps\cdot(p(p-1)/2) \equiv p \pmod {p^2}
    \]
    e o resultado segue também.

    Para o passo indutivo, suponha que o resultado vale para $\beta \geq 1$ e seja $n = p^{\beta + 1}$. Então,
    \[
        a^{n} - 1 = a^{p^{\beta + 1}} - 1 = (a^p)^{p^{\beta}} - 1,
    \]
    por indução com os parâmetros $(a = a^p)$ e $(n = p^\beta)$, temos
    \[
        \nu_p(a^n - 1) = \nu_p(a^p - 1) + \nu_p(p^\beta) = \nu_p(a-1) + 1 + \beta = \nu_p(a-1) + \nu_p(p^{\beta + 1}),
    \]
    o que prova a afirmação.

    Já temos o suficiente para o caso geral, suponha que $\nu_p(a-1) =\alpha \geq 1$ e $\nu_p(n) = \beta$, de onde $n = p^\beta \cdot k$ com $(p,k) = 1$. Então
    \[
        a^n - 1 = (a^{p^\beta})^k - 1 = (a^{p^\beta} - 1) \cdot \bigg(\sum_{j = 0}^{k-1} (a^{p^\beta})^j\bigg),
    \]
    já sabemos que $\nu_p((a^{p^\beta} - 1)) = \alpha + \beta$, então basta mostrar que o somatório não é divisivel por $p$. Notamos, pelo teorema de Fermat,
    que para qualquer $\beta \geq 1$,
    \[
        a^{p^\beta} = (a^p)^{p^{\beta - 1}} \equiv a^{p^{\beta - 1}} \pmod p,
    \]
    ou seja $a^{p^\beta} \equiv a \pmod p$, mas $a \equiv 1 \pmod p$ pois $\nu_p(a-1) \geq 1$. No somatório, isso se traduz como
    \[
        \sum_{j = 0}^{k-1} (a^{p^\beta})^j \equiv \sum_{j=0}^{k-1} 1 \equiv k \not \equiv 0 \pmod p,
    \]
    pois $(k,p) = 1$. Isto finaliza a demonstração.
\end{proof}

A prova do segundo item é quase que idêntica a do primeiro, só fazemos uso da outra fatoração usual. Serei um pouco mais sucinto.

\begin{proof}
    \textbf{(b)} Caso $\nu_p(n) = \beta = 0$ e $\nu_p(a + 1) = \alpha \geq 1$. Escrevemos 
    \[
        a^n + 1 = (a + 1) \cdot \bigg( \sum_{j = 0}^{n-1} (-1)^j a^j \bigg) \Rightarrow \nu_p(a^n + 1) = \alpha + \nu_p\bigg( \sum_{j = 0}^{n-1} (-1)^j a^j \bigg)
    \]
    como $\alpha \geq 1$, $a \equiv -1 \pmod p$, ou seja 
    \[
        \sum_{j = 0}^{n-1} (-1)^j a^j \equiv \sum_{j=0}^{n-1} (-1)^{2j} \equiv n \not \equiv 0 \pmod p
    \]
    e portanto $\nu_p\big( \sum_{j = 0}^{n-1} (-1)^j a^j\big) = 0$.

    Caso $n = p$, $\nu_p(a+1) = \alpha \geq 1$. Como no caso anterior, basta mostrar que $\nu_p\big( \sum_{j = 0}^{p-1} (-1)^j a^j\big) = 1$.
    Escrevemos $a = p^\alpha s - 1$ com $(p,s) = 1$. Substituindo no somatório,
    \[
        \sum_{j = 0}^{p-1} (-1)^j a^j = \sum_{j = 0}^{p-1} (-p^\alpha s + 1)^j \equiv p + p^\alpha s \cdot (p(p-1)/2) \equiv p + p^{\alpha + 1}(p-1)/2 \pmod{p^{2\alpha}}  
    \]
    como $\alpha \geq 1$ e $p$ é ímpar, tomando a última equivalência módulo $p^2$ vemos que $\big( \sum_{j = 0}^{p-1} (-1)^j a^j \big) \equiv p \pmod{p^2}$,
    e isso nos dá o resultado que queríamos.

    Caso $n = p^{\beta + 1}$ com $\beta \geq 1$, $\nu_p(a + 1) = \alpha \geq 1$. Exatamente como antes, suponha, por indução, que o resultado é válido para $n = p^\beta$, segue que 
    \[
        \nu_p(a^n + 1) = \nu_p\bigg((a^{p})^{p^\beta} + 1 \bigg) = \nu_p(a^p + 1) + \nu_p(p^{\beta}) = \alpha + 1 + \beta.
    \]
    onde usamos o caso anterior da prova na última igualdade.

    Caso $n = p^\beta k$ com $(k,p) = 1$ e $\beta \geq 1$, $\nu_p(a+1) = \alpha \geq 1$. Escrevemos (como no item anterior),
    \[
        a^n + 1 = (a^{p^\beta} + 1) \cdot \bigg(\sum_{j=0}^{k-1} (-a^{p^\beta})^j \bigg).
    \]
    Pelo caso indutivo, já sabemos que $\nu_p(a^{p^\beta} + 1) = \alpha + \beta$, basta mostrar que $\sum_{j=0}^{k-1} (-a^{p^\beta})^j \not \equiv 0 \pmod p$.
    Mas, pela mesma observação de antes, se $\beta \geq 1$, $a^{p^\beta} \equiv a \pmod p$ e, como $a \equiv -1 \pmod p$, temos 
    \[
        \sum_{j=0}^{k-1} (-a^{p^\beta})^j \equiv \sum_{j=0}^{k-1} (-1 \cdot -1)^j \equiv k \not \equiv 0 \pmod p.
    \]
    O que completa a demosntração.
\end{proof}


\begin{problem}
    \begin{enumerate}[label=(\alph*)]
        \item Prove que $\text{ord}_{2^k} 5 = 2^{k-2}$, para todo $k \geq 2$.
        \item Prove que se $a$ é um inteiro ímpar e $k \geq 2$ então existem $\eps_j \in \{-1,1\}$ e $j \in \mathbb{Z}$ com $0 \leq j < 2^{k-2}$, únicamente determinados,
        tais que $a \equiv \eps_j \cdot 5^j \pmod {2^k}$.
    \end{enumerate}
\end{problem}

\begin{proof}
    \textbf{(a)} Vamos provar por indução em $k$. O resultado é claro para $k = 2$ pois $5 \equiv 1 \pmod 4$. Suponha que vale para $k \geq 2$, vamos provar para $k+1$. Isto é,
    queremos mostrar que $t = \text{ord}_{2^{k+1}} 5 = 2^{k-1}$, sabemos que 
    \[
        \text{ord}_{2^k} 5 \mid t = \text{ord}_{2^{k+1}} 5 \mid \varphi(2^{k+1}) = 2^{k} \Rightarrow t \in \{2^{k-2}, 2^{k-1}, 2^{k}\}.
    \]
    Como $5$ não é raiz primitiva módulo $4$, não pode ser raiz primitiva módulo $2^k$ para $k \geq 2$. Logo $t \in \{2^{k-2}, 2^{k-1}\}$ e
    basta mostrar que $5^{2^{k-2}} \not \equiv 1 \pmod {2^{k+1}}$. Para isso, vamos calcular $\nu_2(5^{2^{k-2}} - 1)$. Vamos usar uma fatoração esperta,
    repetindo o fato que $x^2 - 1 = (x+1)(x-1)$, temos
    \begin{equation}
        \label{lista1:eq:fatoracao_pot_2}
        (5^{2^{k-2}} - 1) = (5^{2^{k-3}} + 1)(5^{2^{k-4}} + 1)\dots (5^{2} + 1)(5 + 1)(5 - 1) = 4 \cdot \prod_{j=0}^{k-3} (5^{2^j} + 1). 
    \end{equation}
    Como $5 \equiv 1 \pmod 4$, para qualquer número par $s$, $5^s \equiv 1 \pmod 4$ e portanto $5^s + 1 \equiv 2 \pmod 4$. Em particular,
    $\nu_2(5^s + 1) = 1$. Usando esse fato na expressão acima, temos 
    \[
        \nu_2(5^{2^{k-2}} - 1) = \nu_2(4) + \sum_{j=0}^{k-3}\nu_2(5^{2^j} + 1) = 2 + k-2 = k.
    \]
    Portanto $2^{k+1} \nmid 5^{2^{k-2}} - 1$, ou seja $5^{2^{k-2}} \not \equiv 1 \pmod {2^{k+1}}$.
\end{proof}


\begin{proof}
    \textbf{(b)} Essa prova é um belo problema de contagem. Note que como $5 \equiv 1 \pmod 4$, para todo $k$, $5^k \equiv 1 \pmod 4$. No entanto, para $k \geq 2$,
    há exatamente $2^{k}/4$ classes de equivalência módulo $2^k$ que são congruentes a $1$ módulo 4. Como $\text{ord}_{2^k} 5 = 2^{k-2} = 2^{k}/4$,
    segue que 
    \[
        \{ \bar{5}^k \pmod {2^k}: 0 \leq k \leq \text{ord}_{2^k} 5 = 2^{k-2}\} = \{\bar{a} \pmod {2^k} : a \equiv 1 \pmod 4\}.  
    \]
    Particularmente, se $a \equiv 1 \pmod 4$, então existe um único $0 \leq j \leq 2^{k-2}$ tal que $5^j \equiv a \pmod{2^k}$. Caso $a \equiv -1 \pmod 4$,
    então existe um único $0\leq j \leq 2^{k-2}$ tal que $5^k \equiv -a \pmod {2^k}$, logo $a \equiv -5^j \pmod {2^k}$. Note que os pares $(\eps_j,j)$ estão únicamente determinados 
    pois, se $\eps_i \neq \eps_j$ então $\eps_i 5^i \not \equiv \eps_j 5^j \pmod 4$ e se $\eps_i = \eps_j$, então $5^i \equiv 5^j$, mas $0 \leq i \neq j <\text{ord}_{2^k} 5$ o que é absurdo.
\end{proof}

\begin{problem}
    Qual é o menor natural $n$ para o qual existe $k$ natural de modo que os 2026 últimos dígitos na representação decimal de $n^k$ são iguais a $1$?
\end{problem}

\begin{proof}
    Esse problema é cabuloso e não fui capaz de resolvê-lo sozinho, nem mesmo com dicas - a solução escrita aqui segue a da \href{https://www.obm.org.br/content/uploads/2017/01/eureka38.pdf}{revista Eureka}.

    Note que a questão se resume a achar o menor $n > 0$ tal que existe $k$ satisfazendo
    \[
        n^k \equiv \sum_{j=0}^{2025} 10^j \equiv \frac{10^{2026} - 1}{9} \equiv -1\cdot 9^{-1}\pmod{10^{2026}}, 
    \]
    equivalentemente, usando o Teorema Chinês dos Restos,
    \begin{align*}
        n^k \equiv -9^{-1} \pmod{2^{2026}}\\
        n^k \equiv -9^{-1} \pmod{5^{2026}}.\\
    \end{align*}
    A ideia da prova é inicialmente restringir $n$, fazemos isso olhando para congruências simples.
    Vamos olhar módulo potências de 2. Como $n^k$ deve terminar com $1$, temos que $n$ tem que ser ímpar. Olhando a primeira congruência módulo 8, temos
    \[
    n^k \equiv -1 \pmod 8,
    \]
    mas como todo número ímpar ao quadrado é $1$ módulo 8, devemos ter que $n \equiv -1 \pmod 8$ e $k \equiv 1 \pmod 2$ ou seja $k = 2l + 1$.
    Olhando módulo $16$, segue que $n$ é congruente a $7$ ou $15$. Mas, se $n$ fosse $\overline{15}$, teríamos
    $15^{2l + 1} \equiv -1 \equiv -9^{-1} \pmod {16}$, o que é absurdo pois $-1 \cdot -9 \equiv 9 \not \equiv 1 \pmod {16}$. Portanto $n \equiv 7 \pmod {16}$ e 
    essa é nossa primeira congruência de importância.

    Olhando módulo $5$, vemos que $n^{2l + 1} \equiv - (-1)^{-1} \equiv 1 \pmod 5$, logo $n \equiv 1 \pmod 5$. Isso segue, pois $-1^{2l + 1} \equiv -1$ e ${-3}^{2l+1} \equiv 2^{2l + 1} \in \{-2,2\}$
    portanto nem $2$ e nem $3$ podem são tais que elevados a um ímpar dão $-1 \pmod 5$.

    Sabemos então que $n \equiv 1 \pmod 5$ e que $n \equiv 7 \pmod {16}$, resolvendo o sistema, temos que $n \equiv 71 \pmod {80}$ e aqui devemos fazer um salto de fé e sonhar que 
    $71$ seja solução.

    Vamos tentar cálcular $t_5 := \text{ord}_{5^{2026}} 71$ usando o problema anterior. Notamos que $t_5 \mid \varphi(5^{2026}) = 4\cdot5^{2025}$. Uma boa estratégia
    é entender o valor de $\nu_5(71^s - 1)$, se este for maior ou igual a $2026$ temos que $71^s \equiv 1 \pmod {5^{2026}}$. Pelo primeiro item do exercício anterior [\ref{prob:1.3.a}],
    temos 
    \[
    \nu_5(71^s - 1) = \nu_5(s) + \nu_5(71 - 1) = \nu_5(s) + 1,
    \]
    portanto, se $s = 5^{2025}$, vemos que
    \[
        \nu_5(71^{5^{2025}} - 1) = 2026 \Rightarrow 71^{5^{2025}} \equiv 1 \pmod {5^{2026}},
    \]
    mas, crucialmente, se $s = 5^x$ com $x < 2025$, então $\nu_5(71^{5^{x}} - 1) = x + 1 < 2026$, e portanto $t_5 \neq 5^x$.
    Isso significa que temos $t_5 = 5^{2025}$.

    Para calcular $t_2 := \text{ord}_{2^{2026}} 71$ usaremos a mesma fatoração que no exercício anterior [\ref{lista1:eq:fatoracao_pot_2}]. Como $t_2 \mid \varphi(2^{2026}) = 2^{2025}$,
    $t_2$ é da forma $2^x$ para algum $x \geq 1$. Vamos calcular $\nu_2(71^{2^x} - 1)$ usando a fatoração. Note que
    \begin{align*}
        (71^{2^{x}} - 1) = (71 - 1) \cdot \prod_{j=0}^{x-1} (71^{2^j} + 1) = (71 - 1)\cdot(71+1)\cdot\prod_{j=1}^{x-1} (71^{2^j} + 1) \\
        \nu_2(71^{2^{x}} - 1) = \nu_2(70) + \nu(72) + \sum_{j=1}^{x-1}\nu_2(71^{2^x} + 1)
    \end{align*}

    Como $71 \equiv -1 \pmod 8$, para qualquer $x \geq 1$ vale que $71^{2^x} = (-1^2)^{2^{x-1}} \equiv 1 \pmod 8$, ou seja $71^{2^x} + 1 \equiv 2 \pmod 8$. Portanto
    $\nu_2(71^{2^j} + 1) = 1$. Substituindo na equação acima temos:
    \[
        \nu_2(71^{2^{x}} - 1) = 1 + 3 + x-1 = x + 3.
    \]
    Assim como antes, segue que $71^{2^{2023}} \equiv 1 \pmod {2^{2026}}$, mas $71^{2^{x}} \not \equiv 1 \pmod {2^{2026}}$ para qualquer $x \leq 2022$. Logo sabemos que
    $t_2 = 2^{2023}$.

    Agora estamos quase prontos. Como $71 \equiv 1 \pmod 5$, $t_5 = 5^{2025}$ e há exatamente $5^{2025}$ números $0 \leq m < 5^{2026}$ com $m \equiv 1 \pmod 5$,
    segue que para cada um deles, existe $0\leq s < t_5$ com $71^{s} \equiv m \pmod {5^{2025}}$. Em particular,
    como $\frac{-1}{9} \equiv 1 \pmod 5$, segue que existe $0 \leq k_5 < t_5$ com
    \[
        71^{k_5} \equiv -(9)^{-1} \pmod {5^{2026}}.
    \]

    Da mesma forma, $71 \equiv 7 \pmod {16}$, portanto $71^x \equiv \{1,7\} \pmod {16}$. Mas existem exatamente $2^{2023}$ números $0 \leq m < 2^{2026}$ com $m \equiv \{1,7\} \pmod {16}$,
    portanto para cada um deles existe um único $0 \leq s < t_2$ com $71^s \equiv m \pmod{2^{2026}}$. Em particular, como $-1 \equiv 7\cdot9 \pmod {16}$, segue que $\frac{-1}{9} \equiv 7 \pmod{16}$
    e existe $0 \leq k_2 < t_2$ com 
    \[
        71^{k_2} \equiv -(9)^{-1} \pmod {2^{2026}}.
    \]

    Como $(t_2,t_5) = (2^{2023}, 5^{2025}) = 1$, pelo teorema chinês dos restos, existe $k > 0$ natural grande satisfazendo 
    \begin{align*}
        71^k > 10^{2027}\\
        k \equiv k_2 \pmod{t_2}\\
        k \equiv k_5 \pmod{t_5}
    \end{align*}
    E por fim (graças a deus), 
    \begin{align*}
        71^{k} = 71^{k_2 + q\cdot t_2} \equiv -9^{-1} \pmod {2^{2026}}\\
        71^{k} = 71^{k_5 + r\cdot t_5} \equiv -9^{-1} \pmod {5^{2026}}\\
        71^{k} \equiv \frac{10^{2026} - 1}{9} \equiv \underbrace{111\dots1}_{2026} \pmod{10^{2026}}
    \end{align*}
    Portanto, $n = 71$.
     

\end{proof}

\begin{problem}
    O símbolo de Legendre $\big(\frac{a}{p}\big)$ pode ser estendido para o símbolo de Jacobi $\big(\frac{a}{n}\big)$, que está definido para $a$ inteiro 
    arbitrário e $n$ inteiro positivo ímpar por $\big(\frac{a}{n}\big) = \big(\frac{a}{p_1}\big)^{\alpha_1} \dots \big(\frac{a}{p_k}\big)^{\alpha_k}$ se 
    $n = p_1^{\alpha_1}\dots p_k^{\alpha_k}$ é a fatoração prima de $n$ (onde os $\big(\frac{a}{p_j}\big)$ são dados pelo símbolo de Legendre usual); temos 
    $\big(\frac{a}{1}\big) = 1$ para todo inteiro $a$.

    Prove as seguintes propriedades do símbolo de Jacobi, que podem ser usadas para calcular rapidametne símbolos de Legendre (e de Jacobi):
    \begin{enumerate}
        \item Se $a \equiv b \mod n$ então $\big(\frac{a}{n}\big) = \big(\frac{b}{n}\big)$.
        \item $\big(\frac{a}{n}\big) = 0$ se $\gcd(a,n) \neq 1$ e $\big(\frac{a}{p}\big) \in \{-1,1\}$ se $\gcd(a,n) = 1$.
        \item $\big(\frac{ab}{n}\big) = \big(\frac{a}{n}\big)\big(\frac{b}{n}\big)$; em particular, $\big(\frac{a^2}{n}\big) \in \{0,1\}$.
        \item $\big(\frac{a}{mn}\big) = \big(\frac{a}{n}\big)\big(\frac{a}{m}\big)$; em particular, $\big(\frac{a}{n^2}\big) \in \{0,1\}$.
        \item Se $m$ e $n$ são positivos e ímpares, então $\big(\frac{m}{n}\big) = (-1)^{(m-1)/2 \cdot (n-1)/2} \big(\frac{n}{m}\big)$.
        \item $\big(\frac{-1}{n}\big) = (-1)^{(n-1)/2}$.
        \item $\big(\frac{2}{n}\big) = (-1)^{(n^2 - 1)/8}$ se $n$ é ímpar.
    \end{enumerate}
\end{problem}


\begin{proof}
\textbf{(1)} Note que se $a \equiv b \pmod n$, então $a \equiv b \pmod{p_j}$ para todo $1 \leq j \leq k$. Pela propriedade usual do símbolo de Legendre,
$\big( \frac{a}{p_j} \big) = \big( \frac{b}{p_j}\big)$ para todo $j$ e, portanto,
\[
    \bigg(\frac{a}{n}\bigg) = \prod_{j=1}^{r}\bigg( \frac{a}{p_j} \bigg)^{\alpha_j} = \prod_{j=1}^{r}\bigg( \frac{b}{p_j} \bigg)^{\alpha_j} = \bigg(\frac{b}{n}\bigg).
\]

\textbf{(2)} Se $(a,n) \neq 1$, então existe algum primo $p_i$ tal que $p_i \mid a$, portanto $\big(\frac{a}{p_i}\big) = 0$ e 
\[ \prod_{j=1}^{r}\bigg( \frac{a}{p_j} \bigg)^{\alpha_j} = 0.\]
Por outro lado, se $(a,n) = 1$, então para todos os primos $p_i$, temos que $p_i \nmid a$ e temos $\big(\frac{a}{p_i}\big) \in \{-1,1\}$. Portanto
\[
    \prod_{j=1}^{r}\bigg( \frac{a}{p_j} \bigg)^{\alpha_j} \in \{-1,1\}.
\]

\textbf{(3)}
Basta abrir a conta e usar a propriedade dos símbolos usuais de Legendre,
\[
\bigg(\frac{ab}{n}\bigg) = \prod_{j=1}^{r}\bigg( \frac{ab}{p_j} \bigg)^{\alpha_j} = \prod_{j=1}^{r}\bigg( \frac{a}{p_j} \bigg)^{\alpha_j}\bigg( \frac{b}{p_j} \bigg)^{\alpha_j} =
\prod_{j=1}^{r}\bigg( \frac{a}{p_j} \bigg)^{\alpha_j} \prod_{j=1}^{r} \bigg( \frac{b}{p_j} \bigg)^{\alpha_j} =  
\bigg(\frac{a}{n}\bigg)\bigg(\frac{b}{n}\bigg).
\]

\textbf{(4)} Sejam $q_1 \dots q_r$ os primos que dividem $n$ ou $m$. Escrevemos $n = q_1^{\alpha_1}\dots q_r^{\alpha_r}$ e $m = q_1^{\beta_1} \dots q_r^{\beta_r}$
onde os $\alpha_i$ e $\beta_j$ podem potencialmente ser $0$. Temos
$$nm = q_1^{\alpha_1 + \beta_1}\dots q_r^{\alpha_r + \beta_r}$$
e logo
$$\bigg( \frac{a}{nm} \bigg) = \prod_{j = 1}^{r} \bigg(\frac{a}{q_j}\bigg)^{\alpha_j + \beta_j} = \prod_{j = 1}^{r} \bigg(\frac{a}{q_j}\bigg)^{\alpha_j} \prod_{j = 1}^{r} \bigg(\frac{a}{q_j}\bigg)^{\beta_j}.$$
Agora notamos que se $q_j \nmid n$, então $\alpha_j = 0$ e  se $q_i \nmid m$, então $\beta_i = 0$, então os produtórios acima se expressam como
\[
    \bigg( \frac{a}{nm} \bigg) =  \prod_{q_j \mid n} \bigg(\frac{a}{q_j}\bigg)^{\alpha_j} \prod_{q_i \mid m} \bigg(\frac{a}{q_i}\bigg)^{\beta_i} = \bigg(\frac{a}{n}\bigg) \bigg(\frac{a}{m}\bigg).
\]

\textbf{(5)} Esse é mais interessante, vamos usar reciprocidade quadrática e as propriedades anteriores. Primeiramente, note que por \textbf{(2)}, a fórmula é válida se $(m,n) \neq 1$,
já que tanto $\big(\frac{m}{n}\big) = 0$ quanto $\big(\frac{n}{m}\big) = 0$. Podemos supor então que $(m,n) = 1$. Outro caso de interesse é que se $a^2 \mid m$, então $m = a^2m'$ e por \textbf{(3)},
$\big(\frac{m}{n}\big) = \big(\frac{m'}{n}\big)$. Já que o mesmo vale para o "denominador" do símbolo de Legendre, podemos supor ainda mais que $m$ e $n$ são livres de quadrados. Ou seja, podemos considerar (ad hoc) que suas fatorações 
são $n = p_1\dots p_l \cdot r_1 \dots r_k $ e $m = q_1 \dots q_t \cdot s_1 \dots s_h$ onde os $p_i \equiv q_j \equiv 1 \pmod 4$, os $r_i \equiv s_j \equiv 3 \pmod 4$ e os primos das fatorações são todos distintos.

Após todas nossas suposições, temos (usando a propriedade \textbf{(3)} e \textbf{(4)}  várias vezes)
\[
    \bigg(\frac{m}{n}\bigg) = \bigg(\frac{q_1 \dots q_t \cdot s_1 \dots s_h}{p_1\dots p_l \cdot r_1 \dots r_k}\bigg)
    = \bigg(\frac{q_1 \dots q_t}{p_1\dots p_l}\bigg)\bigg(\frac{s_1 \dots s_h}{p_1\dots p_l}\bigg) \bigg(\frac{q_1 \dots q_t}{r_1 \dots r_k}\bigg)\bigg(\frac{s_1 \dots s_h}{r_1 \dots r_k}\bigg),
\]
ou seja,
\[
    \bigg(\frac{m}{n}\bigg) = \prod_{(q_i,p_j)} \bigg(\frac{q_i}{p_j}\bigg) \cdot \prod_{(s_i,p_j)} \bigg(\frac{s_i}{p_j}\bigg) \cdot \prod_{(q_i,r_j)} \bigg(\frac{q_i}{r_j}\bigg) \cdot \prod_{(s_i,r_j)} \bigg(\frac{s_i}{r_j}\bigg).
\]
Pela lei da reciprocidade quadrática, se $h$ é um primo com $h \equiv 1 \pmod 4$ e $g$ é outro primo qualquer, então $\big(\frac{h}{g}\big) = \big(\frac{g}{h}\big)$ e se ambos $g$ e $h$ forem congruentes a 3 módulo 4, então 
$\big(\frac{h}{g}\big) = -\big(\frac{g}{h}\big)$. Podemos usar isso na expressão acima para obter
\[
    \bigg(\frac{m}{n}\bigg) = \prod_{(q_i,p_j)} \bigg(\frac{p_j}{q_i}\bigg) \cdot \prod_{(s_i,p_j)} \bigg(\frac{p_j}{s_i}\bigg) \cdot \prod_{(q_i,r_j)} \bigg(\frac{r_j}{q_i}\bigg) \prod_{(s_i,r_j)} - \bigg(\frac{r_j}{s_i}\bigg),
\]
de forma que (juntando os produtórios)
\[
    \bigg(\frac{m}{n}\bigg) = (-1)^{kh} \bigg(\frac{n}{m}\bigg).
\]
Para o resultado, basta mostrar que $kh \equiv (n-1)/2 \cdot (m-1)/2 \pmod 2$ (note que são inteiros uma vez que $n$ e $m$ são ímpares). Vamos olhar para $n$ e $m$ módulo 4, observamos que 
\[
    n \equiv p_1\dots p_l \cdot r_1 \dots r_k \equiv r_1 \dots r_k \equiv 3^k \equiv \begin{cases}
        1 &\text{se } k \equiv 0 \mod 2\\
        3 &\text{se } k \equiv 1 \mod 2
    \end{cases}\bigg\} \pmod 4
\]
o resultado análogo segue para $m$ e $h$. Disso já obtemos que se $h$ ou $k$ forem pares, então $n$ ou $m$ são 1 módulo 4, portanto $(n-1)/2$ ou $(m-1)/2$ é par
e $hk \equiv 0 \equiv (n-1)/2 \cdot (m-1)/2 \pmod 2$. Se ambos $h$ e $k$ forem ímpares, então $n \equiv m \equiv 3 \pmod 4$, logo $(n-1)/2$ e $(m-1)/2$ são ímpares
e $hk \equiv 1 \equiv (n-1)/2 \cdot (m-1)/2 \pmod 2$ concluindo a demonstração.

\textbf{(6)} Vamos fazer uma análise semelhante a \textbf{(5)}. Pelo observado anteriormente, podemos supor que $n$ é livre de quadrados e se escreve
$n = p_1 \dots p_l \cdot r_1 \dots r_k$ com os $p_i \equiv 1 \pmod 4$ e $r_i \equiv 3 \pmod 4$. Abrindo o símbolo de Jacobi, temos então
\[
    \bigg(\frac{-1}{n}\bigg) = \prod_{p_i} \bigg(\frac{-1}{p_i}\bigg) \prod_{r_j} \bigg(\frac{-1}{r_j}\bigg).
\]
Como $\big(\frac{-1}{x}\big) = 1$ se $x$ é primo e $x \equiv 1 \pmod 4$ e $\big(\frac{-1}{x}\big) = -1$ se $x$ for um primo com $x \equiv 3 \pmod 4$, segue que 
\[
    \bigg(\frac{-1}{n}\bigg) = (-1)^{k}
\] 
ou seja, para mostrar a igualdade, basta veriricar que $k \equiv (n-1)/2 \pmod 2$ e já fizemos isso na prova da propriedade anterior.

\textbf{(7)} Seguindo a mesma ideia, vamos fatorar $n$ de maneira esperta. Vimos que, sem perda de generalidade, podemos supor $n$ livre de quadrados, então escrevemos a fatoração prima de $n$ como
\[
    n = (p_1^+p_2^+\dots p_l^+) \cdot (p_1^-p_2^-\dots p_k^-) \cdot (q_1^+q_2^+\dots q_r^+) \cdot (q_1^-q_2^-\dots q_s^-)
\]
onde cada $p_i^+ \equiv 1 \pmod 8$, $p_i^- \equiv -1 \pmod 8$, $q_i^+ \equiv 3 \pmod 8$ e $q_i^- \equiv -3 \pmod 8$. Usando a propriedade \textbf{(4)}, temos 
\[
    \bigg(\frac{2}{n}\bigg) = \prod_{p_i^+} \bigg(\frac{2}{p_i^+}\bigg) \cdot \prod_{p_i^-} \bigg(\frac{2}{p_i^-}\bigg) \cdot \prod_{q_i^+} \bigg(\frac{2}{q_i^+}\bigg) \cdot \prod_{q_i^-} \bigg(\frac{2}{q_i^-}\bigg).
\]
Por reciprocidade quadrática, sabemos que para todo $i$ vale $\big(\frac{2}{p_i^+}\big) = \big(\frac{2}{p_i^-}\big) = 1$ e $\big(\frac{2}{q_i^+}\big) = \big(\frac{2}{q_i^-}\big) = -1$, portanto, a equação acima reduz-se para
\[
    \bigg(\frac{2}{n}\bigg) = (-1)^{r + s}.
\]
Para finalizar a demonstração, basta mostrar que $r+s \equiv (n^2-1)/8 \pmod 2$ ou, equivalentemente, desejamos mostrar 
\[
    r + s \equiv 0 \pmod 2 \iff n \equiv \{-1,1\} \pmod 8 \quad \text{e} \quad r + s \equiv 1 \pmod 2 \iff n \equiv \{-3,3\} \pmod 8.
\]
Notamos primeiramente que
\[
    n = (p_1^+p_2^+\dots p_l^+) \cdot (p_1^-p_2^-\dots p_k^-) \cdot (q_1^+q_2^+\dots q_r^+) \cdot (q_1^-q_2^-\dots q_s^-) \equiv (1)^l \cdot (-1)^k \cdot (3)^r \cdot (-3)^s \pmod 8,
\]
ou seja, $n \equiv \eps  \cdot (3)^r \cdot (-3)^s \pmod 8$ onde $\eps \in \{-1,1\}$. Agora para a análise de casos. Se $r + s$ for par, então ou $r$ e $s$ são pares 
onde $n \equiv \eps  \cdot 1 \cdot 1 \in \{-1,1\} \pmod 8$ ou $r$ e $s$ são ímpares e temos $n \equiv \eps  \cdot 3 \cdot -3 \equiv -\eps \in \{-1,1\} \pmod 8$. Se, por outro lado,
$r + s$ for ímpar, então ou $r$ é ímpar e $s$ é par onde $n \equiv \eps \cdot 3 \cdot 1 \in \{3,-3\} \pmod 8$ ou $r$ é par e $s$ é ímpar, onde também temos $n \equiv \eps \cdot 1 \cdot -3 \in \{3,-3\} \pmod 8$. O que 
conclui a demonstração.



\end{proof}
