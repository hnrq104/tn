\pagebreak
\section{Lista 1 - \date{12/1/2026}}

\begin{problem}
    Dados inteiros positivos $a, b$ e $c$, dois a dois primos entre si, demonstre que $2abc - ab - bc -ca$ é o 
    maior número inteiro que não pode expressar-se na forma $xbc + yca + zab$ com $x,y$ e $z$ inteiros não negativos.
\end{problem}

\begin{proof}
    Note que como $(b,c) = 1$, temos que $(ab,ac) = a$ e, portanto por Bachét-Bezout existe solução
    para $z'ab + y'ca = a$ com $z',y'$ inteiros. Por sua vez, como $(a,bc) = 1$, existe solução para 
    $ma + nbc = 1$ com $m,n$ inteiros. Juntando as duas equações, encontramos $mz'ab + my'ca + nbc = 1$ que é solução para 
    a equação $xbc + yca + zab = 1$ e, portanto, temos soluções para $xbc + yca + zab = k$ para qualquer inteiro $k$.

    Vamos mostrar que $2abc - ab - bc -ca$ não pode ser escrito como $xbc + yca + zab$ para $x,y,z \in \N$. Suponha, que conseguimos,
    temos 
    \begin{align*}
        2abc - ab - bc -ca = xbc + yca + zab\\
        2abc = (x+1)bc + (y+1)ca + (z+1)ab 
    \end{align*}
    tomando a segunda equação módulo $a$, achamos
    \[
        0 \equiv (x+1)bc \pmod a \Rightarrow x+1 \equiv 0 \pmod a
    \]
    ou seja, $a \mid (x+1)$. Como $x \geq 0$, devemos ter $(x+1) \geq a$. Simetricamente (tomando módulo $b$ e depois $c$), sabemos que $(y+1) \geq b$ e $(z+1) \geq c$.
    Mas já encontramos contradição, uma vez que essas desigualdades implicam
    \[
    (x+1)bc + (y+1)ca +(z+1)ab \geq abc + bca + cab = 3abc > 2abc
    \]

    Agora seja $n > 2abc - bc - ac - ab$, mostraremos que existe solução natural para $n = xbc + yac + zab$. Primeiro, vamos
    caracterizar as soluções inteiras, que existem pela observação anterior. Note que se $(x,y,z)$ e $(x',y',z')$ são soluções,
    então
    \begin{equation}
        \label{lista1:eq:prob1}
        (x-x')bc + (y-y')ac + (z-z')ab = 0
    \end{equation}
    tomando a equação módulo $a$, vemos que $(x-x') \equiv 0 \pmod a$ e portanto $x' = x + ra$ para algum $r \in \Z$. Simetricamente,
    vemos que $y' = y + sb$ e $z' = z + tc$ para $s,t \in \Z$. Portanto, [\ref{lista1:eq:prob1}] se expressa como 
    \[
        (ra)bc + (sb)ac + (tc)ab = (r+s+t)abc = 0 \iff (r+s+t) = 0 
    \]
    

\end{proof}

\begin{problem}
    Seja $p$ um número primo ímpar. Seja $s$ o menor inteiro positivo que não é resíduo quadrático módulo $p$.
    \begin{enumerate}[label=(\alph*)]
        \item Mostre que $p > s^2 - s$.
        \item Suponha que $p > 5$ e que $-1$ seja resíduo quadrático módulo $p$: mostre que $p>2s^2 - s$.
    \end{enumerate}
\end{problem}


\begin{proof}
    
\end{proof}


\begin{problem}
    Seja $p$ um primo ímpar, $a$ um inteiro e $n$ um inteiro positivo. Sejam $\alpha$ e $\beta$ inteiros negativos, com $\alpha > 0$. Prove:
    \begin{enumerate}[label=(\alph*)]
        \item Se $p^\beta$ e $p^\alpha$ são as maiores potências de $p$ que dividem $n$ e $a-1$ respectivamente então $p^{\alpha + \beta}$ é 
        a maior potência que divide $a^n - 1$.
        \item Se $n$ é ímpar e $p^\beta$ e $p^\alpha$ são as maiores potências de $p$ que dividem $n$ e $a+1$ respectivamente então $p^{\alpha + \beta}$ é a maior potência
        de $p$ que divide $a^n + 1$.
    \end{enumerate}
\end{problem}


\begin{proof}
    
\end{proof}


\begin{problem}
    \begin{enumerate}[label=(\alph*)]
        \item Prove que $\text{ord}_{2^k} 5 = 2^{k-2}$, para todo $k \geq 2$.
        \item Prove que se $a$ é um inteiro ímpar e $k \geq 2$ então existem $\eps_j \in \{-1,1\}$ e $j \in \mathbb{Z}$ com $0 \leq j \leq 2^{k-2}$, únicamente determinados,
        tais que $a \equiv \eps_j \cdot 5^j \mod 2^k$.
    \end{enumerate}
\end{problem}

\begin{proof}
    
\end{proof}

\begin{problem}
    Qual é o menor natural $n$ para o qual existe $k$ natural de modo que os 2026 últimos dígitos na representação decimal de $n^k$ são iguais a $1$?
\end{problem}


\begin{proof}
    
\end{proof}

\begin{problem}
    O símbolo de Legendre $\big(\frac{a}{p}\big)$ pode ser estendido para o símbolo de Jacobi $\big(\frac{a}{n}\big)$, que está definido para $a$ inteiro 
    arbitrário e $n$ inteiro positivo ímpar por $\big(\frac{a}{n}\big) = \big(\frac{a}{p_1}\big)^{\alpha_1} \dots \big(\frac{a}{p_k}\big)^{\alpha_k}$ se 
    $n = p_1^{\alpha_1}\dots p_k^{\alpha_k}$ é a fatoração prima de $n$ (onde os $\big(\frac{a}{p_j}\big)$ são dados pelo símbolo de Legendre usual); temos 
    $\big(\frac{a}{1}\big) = 1$ para todo inteiro $a$.

    Prove as seguintes propriedades do símbolo de Jacobi, que podem ser usadas para calcular rapidametne símbolos de Legendre (e de Jacobi):
    \begin{enumerate}
        \item Se $a \equiv b \mod n$ então $\big(\frac{a}{n}\big) = \big(\frac{b}{n}\big)$.
        \item $\big(\frac{a}{n}\big) = 0$ se $\gcd(a,n) \neq 1$ e $\big(\frac{a}{p}\big) \in \{-1,1\}$ se $\gcd(a,n) = 1$.
        \item $\big(\frac{ab}{n}\big) = \big(\frac{a}{n}\big)\big(\frac{b}{n}\big)$; em particular, $\big(\frac{a^2}{n}\big) \in {0,1}$.
        \item $\big(\frac{a}{mn}\big) = \big(\frac{a}{n}\big)\big(\frac{a}{m}\big)$; em particular, $\big(\frac{a}{n^2}\big) \in {0,1}$.
        \item Se $m$ e $n$ são positivos e ímpares, então $\big(\frac{m}{n}\big) = (-1)^{(m-1)/2 \cdot (n-1)/2} \big(\frac{n}{m}\big)$.
        \item $\big(\frac{-1}{n}\big) = (-1)^{(n-1)/2}$.
        \item $\big(\frac{2}{n}\big) = (-1)^{(n^2 - 1)/8}$ se $n$ é ímpar.
    \end{enumerate}
\end{problem}


\begin{proof}
    
\end{proof}
