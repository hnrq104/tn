\pagebreak
\section{Lista 1 - \date{12/1/2026}}

\begin{problem}
    Dados inteiros positivos $a, b$ e $c$, dois a dois primos entre si, demonstre que $2abc - ab - bc -ca$ é o 
    maior número inteiro que não pode expressar-se na forma $xbc + yca + zab$ com $x,y$ e $z$ inteiros não negativos.
\end{problem}

\begin{proof}
    Note que como $(b,c) = 1$, temos que $(ab,ac) = a$ e, portanto por Bachét-Bezout existe solução
    para $z'ab + y'ca = a$ com $z',y'$ inteiros. Por sua vez, como $(a,bc) = 1$, existe solução para 
    $ma + nbc = 1$ com $m,n$ inteiros. Juntando as duas equações, encontramos $mz'ab + my'ca + nbc = 1$ que é solução para 
    a equação $xbc + yca + zab = 1$ e, portanto, temos soluções para $xbc + yca + zab = k$ para qualquer inteiro $k$.

    Vamos mostrar que $2abc - ab - bc -ca$ não pode ser escrito como $xbc + yca + zab$ para $x,y,z \in \N$. Suponha, que conseguimos,
    temos 
    \begin{align*}
        2abc - ab - bc -ca = xbc + yca + zab\\
        2abc = (x+1)bc + (y+1)ca + (z+1)ab 
    \end{align*}
    tomando a segunda equação módulo $a$, achamos
    \[
        0 \equiv (x+1)bc \pmod a \Rightarrow x+1 \equiv 0 \pmod a
    \]
    ou seja, $a \mid (x+1)$. Como $x \geq 0$, devemos ter $(x+1) \geq a$. Simetricamente (tomando módulo $b$ e depois $c$), sabemos que $(y+1) \geq b$ e $(z+1) \geq c$.
    Mas já encontramos contradição, uma vez que essas desigualdades implicam
    \[
    (x+1)bc + (y+1)ca +(z+1)ab \geq abc + bca + cab = 3abc > 2abc
    \]

    Agora seja $n > 2abc - bc - ac - ab$, mostraremos que existe solução natural para $n = xbc + yac + zab$. Primeiro, vamos
    caracterizar as soluções inteiras, que existem pela observação anterior. Note que se $(x,y,z)$ e $(x',y',z')$ são soluções,
    então
    \begin{equation}
        \label{lista1:eq:prob1}
        (x-x')bc + (y-y')ac + (z-z')ab = 0
    \end{equation}
    tomando a equação módulo $a$, vemos que $(x-x') \equiv 0 \pmod a$ e portanto $x' = x + ra$ para algum $r \in \Z$. Simetricamente,
    vemos que $y' = y + sb$ e $z' = z + tc$ para $s,t \in \Z$. Portanto, [\ref{lista1:eq:prob1}] se expressa como 
    \[
        (ra)bc + (sb)ac + (tc)ab = (r+s+t)abc = 0 \iff (r+s+t) = 0 
    \]
    Ou seja, se $(x_0, y_0, z_0)$ é uma solução inicial, todas as outras soluções são da forma $(x_0 + ra, y_0 + sb, z_0 + tc)$ onde 
    $r + s + t = 0$, é fácil ver que qualquer qualquer tripla dessa forma também satisfaz a equação original. Nosso problema se resume então
    a encontrar soluções inteiras $(r,s,t)$ para a seguinte série de relações:
    \begin{align*}
        x_0 + ra > -1\\
        y_0 + sb > -1\\
        z_0 + tc > -1\\
        r + s + t = 0
    \end{align*}
    Isolando as variáveis e escrevendo $t$ como $-(r+s)$, temos
    \begin{align*}
        -\frac{(x_0+1)}{a} < r\\
        -\frac{(y_0+1)}{b} < s\\
        r+s < \frac{(z_0 + 1)}{c}
    \end{align*}
    As duas primeiras desigualdades, implicam que 
    \[
       - \bigg(\frac{(x_0+1)}{a} + \frac{(y_0+1)}{b}\bigg) < r+s < \frac{(z_0 + 1)}{c}
    \]
    Notamos (seguindo a resolução do livro para um problema similar) que 
    \[
        \frac{(z_0 + 1)}{c} - - \bigg(\frac{(x_0+1)}{a} + \frac{(y_0+1)}{b}\bigg) = \frac{(z_0 + 1)}{c} + \frac{(x_0+1)}{a} + \frac{(y_0+1)}{b} =
        \frac{n + bc + ac + ab}{abc} > 2
    \]
    pois $n > 2abc - bc - ac - ab$. Segue que o intervalo $\bigg(- \frac{(x_0+1)}{a} - \frac{(y_0+1)}{b}, \frac{(z_0 + 1)}{c}\bigg)$ tem ao menos dois inteiros.
    Particularmente, os números %mais casos aqui
    \[ 
        \bigg\lceil - \frac{(x_0+1)}{a} - \frac{(y_0+1)}{b} \bigg\rceil \quad \text{e} \quad \bigg\lceil - \frac{(x_0+1)}{a} - \frac{(y_0+1)}{b} \bigg\rceil + 1
    \]
    pertencem ao intervalo. Tomando
    \begin{align*}
        r = \begin{cases}
            \ceil{-(x_0+1)/a} & \text{se } -(x_0+1)/a \not\in \Z\\
            \ceil{-(x_0+1)/a} + 1 & \text{se } -(x_0+1)/a \in \Z
        \end{cases}
    \end{align*}
    e $s$ análoga, sendo 
    \begin{align*}
        s = \begin{cases}
            \ceil{-(y_0+1)/b} & \text{se } -(y_0+1)/b \not\in \Z\\
            \ceil{-(y_0+1)/b} + 1 & \text{se } -(y_0+1)/b \in \Z
        \end{cases}
    \end{align*}
    achamos soluções $(r,s,t)$ compatíveis com o sistema de desigualdades.

\end{proof}

\begin{problem}
    Seja $p$ um número primo ímpar. Seja $s$ o menor inteiro positivo que não é resíduo quadrático módulo $p$.
    \begin{enumerate}[label=(\alph*)]
        \item Mostre que $p > s^2 - s$.
        \item Suponha que $p > 5$ e que $-1$ seja resíduo quadrático módulo $p$: mostre que $p>2s^2 - s$.
    \end{enumerate}
\end{problem}


\begin{proof}
    
\end{proof}


\begin{problem}
    Seja $p$ um primo ímpar, $a$ um inteiro e $n$ um inteiro positivo. Sejam $\alpha$ e $\beta$ inteiros negativos, com $\alpha > 0$. Prove:
    \begin{enumerate}[label=(\alph*)]
        \item Se $p^\beta$ e $p^\alpha$ são as maiores potências de $p$ que dividem $n$ e $a-1$ respectivamente então $p^{\alpha + \beta}$ é 
        a maior potência que divide $a^n - 1$.
        \item Se $n$ é ímpar e $p^\beta$ e $p^\alpha$ são as maiores potências de $p$ que dividem $n$ e $a+1$ respectivamente então $p^{\alpha + \beta}$ é a maior potência
        de $p$ que divide $a^n + 1$.
    \end{enumerate}
\end{problem}


\begin{proof}
    
\end{proof}


\begin{problem}
    \begin{enumerate}[label=(\alph*)]
        \item Prove que $\text{ord}_{2^k} 5 = 2^{k-2}$, para todo $k \geq 2$.
        \item Prove que se $a$ é um inteiro ímpar e $k \geq 2$ então existem $\eps_j \in \{-1,1\}$ e $j \in \mathbb{Z}$ com $0 \leq j \leq 2^{k-2}$, únicamente determinados,
        tais que $a \equiv \eps_j \cdot 5^j \mod 2^k$.
    \end{enumerate}
\end{problem}

\begin{proof}
    
\end{proof}

\begin{problem}
    Qual é o menor natural $n$ para o qual existe $k$ natural de modo que os 2026 últimos dígitos na representação decimal de $n^k$ são iguais a $1$?
\end{problem}


\begin{proof}
    
\end{proof}

\begin{problem}
    O símbolo de Legendre $\big(\frac{a}{p}\big)$ pode ser estendido para o símbolo de Jacobi $\big(\frac{a}{n}\big)$, que está definido para $a$ inteiro 
    arbitrário e $n$ inteiro positivo ímpar por $\big(\frac{a}{n}\big) = \big(\frac{a}{p_1}\big)^{\alpha_1} \dots \big(\frac{a}{p_k}\big)^{\alpha_k}$ se 
    $n = p_1^{\alpha_1}\dots p_k^{\alpha_k}$ é a fatoração prima de $n$ (onde os $\big(\frac{a}{p_j}\big)$ são dados pelo símbolo de Legendre usual); temos 
    $\big(\frac{a}{1}\big) = 1$ para todo inteiro $a$.

    Prove as seguintes propriedades do símbolo de Jacobi, que podem ser usadas para calcular rapidametne símbolos de Legendre (e de Jacobi):
    \begin{enumerate}
        \item Se $a \equiv b \mod n$ então $\big(\frac{a}{n}\big) = \big(\frac{b}{n}\big)$.
        \item $\big(\frac{a}{n}\big) = 0$ se $\gcd(a,n) \neq 1$ e $\big(\frac{a}{p}\big) \in \{-1,1\}$ se $\gcd(a,n) = 1$.
        \item $\big(\frac{ab}{n}\big) = \big(\frac{a}{n}\big)\big(\frac{b}{n}\big)$; em particular, $\big(\frac{a^2}{n}\big) \in \{0,1\}$.
        \item $\big(\frac{a}{mn}\big) = \big(\frac{a}{n}\big)\big(\frac{a}{m}\big)$; em particular, $\big(\frac{a}{n^2}\big) \in \{0,1\}$.
        \item Se $m$ e $n$ são positivos e ímpares, então $\big(\frac{m}{n}\big) = (-1)^{(m-1)/2 \cdot (n-1)/2} \big(\frac{n}{m}\big)$.
        \item $\big(\frac{-1}{n}\big) = (-1)^{(n-1)/2}$.
        \item $\big(\frac{2}{n}\big) = (-1)^{(n^2 - 1)/8}$ se $n$ é ímpar.
    \end{enumerate}
\end{problem}


\begin{proof}
\textbf{(1)} Note que se $a \equiv b \pmod n$, então $a \equiv b \pmod{p_j}$ para todo $1 \leq j \leq k$. Pela propriedade usual do símbolo de Legendre,
$\big( \frac{a}{p_j} \big) = \big( \frac{b}{p_j}\big)$ para todo $j$ e, portanto,
\[
    \bigg(\frac{a}{n}\bigg) = \prod_{j=1}^{r}\bigg( \frac{a}{p_j} \bigg)^{\alpha_j} = \prod_{j=1}^{r}\bigg( \frac{b}{p_j} \bigg)^{\alpha_j} = \bigg(\frac{b}{n}\bigg).
\]

\textbf{(2)} Se $(a,n) \neq 1$, então existe algum primo $p_i$ tal que $p_i \mid a$, portanto $\big(\frac{a}{p_i}\big) = 0$ e 
\[ \prod_{j=1}^{r}\bigg( \frac{a}{p_j} \bigg)^{\alpha_j} = 0.\]
Por outro lado, se $(a,n) = 1$, então para todos os primos $p_i$, temos que $p_i \nmid a$ e temos $\big(\frac{a}{p_i}\big) \in \{-1,1\}$. Portanto
\[
    \prod_{j=1}^{r}\bigg( \frac{a}{p_j} \bigg)^{\alpha_j} \in \{-1,1\}.
\]

\textbf{(3)}
Basta abrir a conta e usar a propriedade dos símbolos usuais de Legendre,
\[
\bigg(\frac{ab}{n}\bigg) = \prod_{j=1}^{r}\bigg( \frac{ab}{p_j} \bigg)^{\alpha_j} = \prod_{j=1}^{r}\bigg( \frac{a}{p_j} \bigg)^{\alpha_j}\bigg( \frac{b}{p_j} \bigg)^{\alpha_j} =
\prod_{j=1}^{r}\bigg( \frac{a}{p_j} \bigg)^{\alpha_j} \prod_{j=1}^{r} \bigg( \frac{b}{p_j} \bigg)^{\alpha_j} =  
\bigg(\frac{a}{n}\bigg)\bigg(\frac{b}{n}\bigg).
\]

\textbf{(4)} Sejam $q_1 \dots q_r$ os primos que dividem $n$ ou $m$. Escrevemos $n = q_1^{\alpha_1}\dots q_r^{\alpha_r}$ e $m = q_1^{\beta_1} \dots q_r^{\beta_r}$
onde os $\alpha_i$ e $\beta_j$ podem potencialmente ser $0$. Temos
$$nm = q_1^{\alpha_1 + \beta_1}\dots q_r^{\alpha_r + \beta_r}$$
e logo
$$\bigg( \frac{a}{nm} \bigg) = \prod_{j = 1}^{r} \bigg(\frac{a}{q_j}\bigg)^{\alpha_j + \beta_j} = \prod_{j = 1}^{r} \bigg(\frac{a}{q_j}\bigg)^{\alpha_j} \prod_{j = 1}^{r} \bigg(\frac{a}{q_j}\bigg)^{\beta_j}.$$
Agora notamos que se $q_j \nmid n$, então $\alpha_j = 0$ e  se $q_i \nmid m$, então $\beta_i = 0$, então os produtórios acima se expressam como
\[
    \bigg( \frac{a}{nm} \bigg) =  \prod_{q_j \mid n} \bigg(\frac{a}{q_j}\bigg)^{\alpha_j} \prod_{q_i \mid m} \bigg(\frac{a}{q_i}\bigg)^{\beta_i} = \bigg(\frac{a}{n}\bigg) \bigg(\frac{a}{m}\bigg).
\]

\textbf{(5)} Esse é mais interessante, vamos usar reciprocidade quadrática e as propriedades anteriores. Primeiramente, note que por \textbf{(2)}, a fórmula é válida se $(m,n) \neq 1$,
já que tanto $\big(\frac{m}{n}\big) = 0$ quanto $\big(\frac{n}{m}\big) = 0$. Podemos supor então que $(m,n) = 1$. Outro caso de interesse é que se $a^2 \mid m$, então $m = a^2m'$ e por \textbf{(3)},
$\big(\frac{m}{n}\big) = \big(\frac{m'}{n}\big)$. Já que o mesmo vale para o "denominador" do símbolo de Legendre, podemos supor ainda mais que $m$ e $n$ são livres de quadrados. Ou seja, podemos considerar (ad hoc) que suas fatorações 
são $n = p_1\dots p_l \cdot r_1 \dots r_k $ e $m = q_1 \dots q_t \cdot s_1 \dots s_h$ onde os $p_i \equiv q_j \equiv 1 \pmod 4$, os $r_i \equiv s_j \equiv 3 \pmod 4$ e os primos das fatorações são todos distintos.

Após todas nossas suposições, temos (usando a propriedade \textbf{(3)} e \textbf{(4)}  várias vezes)
\[
    \bigg(\frac{m}{n}\bigg) = \bigg(\frac{q_1 \dots q_t \cdot s_1 \dots s_h}{p_1\dots p_l \cdot r_1 \dots r_k}\bigg)
    = \bigg(\frac{q_1 \dots q_t}{p_1\dots p_l}\bigg)\bigg(\frac{s_1 \dots s_h}{p_1\dots p_l}\bigg) \bigg(\frac{q_1 \dots q_t}{r_1 \dots r_k}\bigg)\bigg(\frac{s_1 \dots s_h}{r_1 \dots r_k}\bigg),
\]
ou seja,
\[
    \bigg(\frac{m}{n}\bigg) = \prod_{(q_i,p_j)} \bigg(\frac{q_i}{p_j}\bigg) \cdot \prod_{(s_i,p_j)} \bigg(\frac{s_i}{p_j}\bigg) \cdot \prod_{(q_i,r_j)} \bigg(\frac{q_i}{r_j}\bigg) \cdot \prod_{(s_i,r_j)} \bigg(\frac{s_i}{r_j}\bigg).
\]
Pela lei da reciprocidade quadrática, se $h$ é um primo com $h \equiv 1 \pmod 4$ e $g$ é outro primo qualquer, então $\big(\frac{h}{g}\big) = \big(\frac{g}{h}\big)$ e se ambos $g$ e $h$ forem congruentes a 3 módulo 4, então 
$\big(\frac{h}{g}\big) = -\big(\frac{g}{h}\big)$. Podemos usar isso na expressão acima para obter
\[
    \bigg(\frac{m}{n}\bigg) = \prod_{(q_i,p_j)} \bigg(\frac{p_j}{q_i}\bigg) \cdot \prod_{(s_i,p_j)} \bigg(\frac{p_j}{s_i}\bigg) \cdot \prod_{(q_i,r_j)} \bigg(\frac{r_j}{q_i}\bigg) \prod_{(s_i,r_j)} - \bigg(\frac{r_j}{s_i}\bigg),
\]
de forma que (juntando os produtórios)
\[
    \bigg(\frac{m}{n}\bigg) = (-1)^{kh} \bigg(\frac{n}{m}\bigg).
\]
Para o resultado, basta mostrar que $kh \equiv (n-1)/2 \cdot (m-1)/2 \pmod 2$ (note que são inteiros uma vez que $n$ e $m$ são ímpares). Vamos olhar para $n$ e $m$ módulo 4, observamos que 
\[
    n \equiv p_1\dots p_l \cdot r_1 \dots r_k \equiv r_1 \dots r_k \equiv 3^k \equiv \begin{cases}
        1 &\text{se } k \equiv 0 \mod 2\\
        3 &\text{se } k \equiv 1 \mod 2
    \end{cases}\bigg\} \pmod 4
\]
o resultado análogo segue para $m$ e $h$. Disso já obtemos que se $h$ ou $k$ forem pares, então $n$ ou $m$ são 1 módulo 4, portanto $(n-1)/2$ ou $(m-1)/2$ é par
e $hk \equiv 0 \equiv (n-1)/2 \cdot (m-1)/2 \pmod 2$. Se ambos $h$ e $k$ forem ímpares, então $n \equiv m \equiv 3 \pmod 4$, logo $(n-1)/2$ e $(m-1)/2$ são ímpares
e $hk \equiv 1 \equiv (n-1)/2 \cdot (m-1)/2 \pmod 2$ concluindo a demonstração.

\textbf{(6)} Vamos fazer uma análise semelhante a \textbf{(5)}. Pelo observado anteriormente, podemos supor que $n$ é livre de quadrados e se escreve
$n = p_1 \dots p_l \cdot r_1 \dots r_k$ com os $p_i \equiv 1 \pmod 4$ e $r_i \equiv 3 \pmod 4$. Abrindo o símbolo de Jacobi, temos então
\[
    \bigg(\frac{-1}{n}\bigg) = \prod_{p_i} \bigg(\frac{-1}{p_i}\bigg) \prod_{r_j} \bigg(\frac{-1}{r_j}\bigg).
\]
Como $\big(\frac{-1}{x}\big) = 1$ se $x$ é primo e $x \equiv 1 \pmod 4$ e $\big(\frac{-1}{x}\big) = -1$ se $x$ for um primo com $x \equiv 3 \pmod 4$, segue que 
\[
    \bigg(\frac{-1}{n}\bigg) = (-1)^{k}
\] 
ou seja, para mostrar a igualdade, basta veriricar que $k \equiv (n-1)/2 \pmod 2$ e já fizemos isso na prova da propriedade anterior.

\textbf{(7)} Seguindo a mesma ideia, vamos fatorar $n$ de maneira esperta. Vimos que, sem perda de generalidade, podemos supor $n$ livre de quadrados, então escrevemos a fatoração prima de $n$ como
\[
    n = (p_1^+p_2^+\dots p_l^+) \cdot (p_1^-p_2^-\dots p_k^-) \cdot (q_1^+q_2^+\dots q_r^+) \cdot (q_1^-q_2^-\dots q_s^-)
\]
onde cada $p_i^+ \equiv 1 \pmod 8$, $p_i^- \equiv -1 \pmod 8$, $q_i^+ \equiv 3 \pmod 8$ e $q_i^- \equiv -3 \pmod 8$. Usando a propriedade \textbf{(4)}, temos 
\[
    \bigg(\frac{2}{n}\bigg) = \prod_{p_i^+} \bigg(\frac{2}{p_i^+}\bigg) \cdot \prod_{p_i^-} \bigg(\frac{2}{p_i^-}\bigg) \cdot \prod_{q_i^+} \bigg(\frac{2}{q_i^+}\bigg) \cdot \prod_{q_i^-} \bigg(\frac{2}{q_i^-}\bigg).
\]
Por reciprocidade quadrática, sabemos que para todo $i$ vale $\big(\frac{2}{p_i^+}\big) = \big(\frac{2}{p_i^-}\big) = 1$ e $\big(\frac{2}{q_i^+}\big) = \big(\frac{2}{q_i^-}\big) = -1$, portanto, a equação acima reduz-se para
\[
    \bigg(\frac{2}{n}\bigg) = (-1)^{r + s}.
\]
Para finalizar a demonstração, basta mostrar que $r+s \equiv (n^2-1)/8 \pmod 2$ ou, equivalentemente, desejamos mostrar 
\[
    r + s \equiv 0 \pmod 2 \iff n \equiv \{-1,1\} \pmod 8 \quad \text{e} \quad r + s \equiv 1 \pmod 2 \iff n \equiv \{-3,3\} \pmod 8.
\]
Notamos primeiramente que
\[
    n = (p_1^+p_2^+\dots p_l^+) \cdot (p_1^-p_2^-\dots p_k^-) \cdot (q_1^+q_2^+\dots q_r^+) \cdot (q_1^-q_2^-\dots q_s^-) \equiv (1)^l \cdot (-1)^k \cdot (3)^r \cdot (-3)^s \pmod 8,
\]
ou seja, $n \equiv \eps  \cdot (3)^r \cdot (-3)^s \pmod 8$ onde $\eps \in \{-1,1\}$. Agora para a análise de casos. Se $r + s$ for par, então ou $r$ e $s$ são pares 
onde $n \equiv \eps  \cdot 1 \cdot 1 \in \{-1,1\} \pmod 8$ ou $r$ e $s$ são ímpares e temos $n \equiv \eps  \cdot 3 \cdot -3 \equiv -\eps \in \{-1,1\} \pmod 8$. Se, por outro lado,
$r + s$ for ímpar, então ou $r$ é ímpar e $s$ é par onde $n \equiv \eps \cdot 3 \cdot 1 \in \{3,-3\} \pmod 8$ ou $r$ é par e $s$ é ímpar, onde também temos $n \equiv \eps \cdot 1 \cdot -3 \in \{3,-3\} \pmod 8$. O que 
conclui a demonstração.



\end{proof}
