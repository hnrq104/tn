\pagebreak
\section{Lista 2 - 26/01/2026}

\begin{problem}
    Sejam $a,n \in \N^*$ e considere a sequência $(x_k)$ definida por $x_1 = a, x_{k+1} = a^{x_k}$ para todo $k \in \N$. Demonstre que existe $N \in \N$ tal que 
    $x_{k+1} \equiv x_k \pmod n$ para todo $k \geq N$.
\end{problem}

\begin{obs}
Note que, se $a > 1$, a sequência $(x_k)$ é estritamente crescente já que $x_{k+1} = a^{x_k} \geq 2^{x_k} > x_k$ e tende para infinito.
\end{obs}

\begin{proof}
    Caso $a = 1$, então o resultado segue pois $x_k = 1$ para todo $k$. Vamos provar os outros casos por indução em $n$.


    Caso $a \neq 1$. Para $n = 1$ é óbvio, uma vez que $x_{k} \equiv 0 \pmod 1$ para todo $k$ independente de $a$.
    Seja $n > 1$ e suponha que vale para todo $m < n$ i.e. existe $N_m$ tal que se $k \geq N_m$, então $x_{k+1} \equiv x_k \pmod m$.
    Fatorando $n$, podemos supor que
    \[
        n = \prod_{i=1}^{r} p_i^{\alpha_i}
    \]
    com os $p_i$ sendo primos distintos. Pelo Teorema Chinês dos Restos, o problema se resume a mostrar que existe $N$ tal que para todo $k \geq N$, vale para todo $i$,
    \[
        x_{k+1} \equiv x_{k} \pmod{p_i^{\alpha_i}}.
    \]
    Vamos encontrar um $N_i$ para cada $p_i$ e tomar $N = \max \,\{N_i : 1 \leq i \leq r\}$.
    Se $(a,p_i) = p_i$, temos $a = p_iq$ e basta que $x_k \geq \alpha_i$ para $x_{k+1} = a^{x_k} = (p_i\cdot q)^{x_k} \equiv 0 \pmod{p_i^{\alpha_i}}$. Como $x_k \to \infty$,
    existe $N_i$ satisfazendo $x_k \equiv 0 \pmod{p_i^{\alpha_i}}$ para todo $k \geq N_i$. Por outro lado, se $(a,p_j) = 1$, podemos
    usar indutivamente o resultado para $m = \varphi(p_j^{\alpha_j}) < n$ de onde segue que existe $M_j$ tal que
    \[
        x_{k+1} \equiv x_{k} \pmod{m = \varphi(p_j^{\alpha_j})} \quad \forall k \geq M_j,
    \]
    portanto
    \[
        x_{k+2} = a^{x_{k+1}} = a^{m\cdot t + x_{k}} \equiv a^{x_k} = x_{k+1} \pmod {p_j^{\alpha_j}} \quad \forall k \geq M_j
    \]
    e portanto, podemos tomar $N_j = M_j + 1$. Tomando $N$ o máximo dos $N_i$, segue que 
    \[
    \forall k \geq N, \forall 1 \leq i \leq r \quad x_{k+1} \equiv x_{k} \pmod{p_i^{\alpha_i}}
    \]
    e, por fim,
    \[
    x_{k+1} \equiv x_{k} \pmod{n} \quad \forall k \geq N
    \]    

\end{proof}

\begin{problem}
    \begin{enumerate}[label=(\alph*)]
        \item Seja $F_n = 2^{2^n} + 1$ o $n$-ésimo número de Fermat. Prove que todo fator primo de $F_n$ 
    é da forma $k\cdot2^{n+1}$.
        \item Prove que, se $n \geq 2$, então todo fator primo de $F_n$ é da forma $k\cdot2^{n+2} + 1$.
        \item Mostre que $2^{2^5} + 1$ é composto.
    \end{enumerate}
\end{problem}

\begin{proof}
    Vamos fazer o item (a) e (b) simultâneamente. Note que, se $p \mid F_n$, então $p$ é ímpar e
    \[
        2^{2^n} \equiv -1 \pmod p \quad \land \quad 2^{2^{n+1}} \equiv 1 \pmod p,
    \]
    logo $t := \text{ord}_p 2 \mid 2^{n+1}$, i.e $t := 2^{q}$ com $1 \leq q \leq 2^{n+1}$. Agora usamos um truque já visto [\ref{lista1:eq:fatoracao_pot_2}], 
    escrevemos
    \begin{equation}
        \label{l2:p2:eq1}
        2^{2^n} + 1 = 2^{2^n} - 1 + 2 = \prod_{m = 0}^{n-1} (2^{2^m} + 1) + 2. 
    \end{equation}
    Vamos provar que $q = n+1$. Se $q \leq n$, sabemos do fato que $(2^{2^{q-1}})^2 \equiv 1 \pmod p$ e $2^{2^{q-1}} \not \equiv 1 \pmod p$
    que $2^{2^{q-1}} \equiv -1 \pmod p$ e, portanto, tomando [\ref{l2:p2:eq1}] módulo $p$, achamos
    \[
    \prod_{m = 0}^{n-1} (2^{2^m} + 1) + 2 \equiv 0 + 2 \not \equiv 0 \pmod p
    \]
    absurdo, pois, por hipótese, $p \mid F_n$. Portanto $q = n+1$ e temos $t = 2^{n+1}$, logo $t = 2^{n+1} \mid \varphi(p) = p - 1$ e segue a letra \textbf{(a)},
    $p = k \cdot 2^{n+1} + 1$. Para a letra \textbf{(b)}, basta perceber que, caso $n \geq 2$, então, pela letra (a), $p \equiv 1 \pmod 8$, ou seja 
    $(\frac{2}{p}) = 1$, logo $2^{(p-1)/2} \equiv 1 \pmod p$ e segue (da mesma forma que antes) que $t = 2^{n+1} \mid (p-1)/2$, ou seja 
    $p = k\cdot2^{n+2} + 1$.

    Para a parte (c), usamos (b) e torcemos que achemos um fator rapidamente.
\end{proof}

\begin{problem}
    Seja $\alpha = [a_0; a_1, a_2, \dots]$ um número real.
    \begin{enumerate}[label=(\alph*)]
        \item Prove que, se $\text{ord} \alpha > 2$ então existe $\lambda > 1$ tal que para infinitos inteiros positivos $n$, temos
        $a_n \geq \lambda^n$.
        \item Prove que $\text{ord}\alpha \geq 1 + \exp(\limsup_{n\to\infty} \frac{\log \log (a_n + 1)}{n})$.
        \item Prove que para todo $c \geq 2$, existe $\alpha \in \R$ tal que $\text{ord}\alpha = 1 + \exp(\limsup_{n\to\infty} \frac{\log \log (a_n + 1)}{n}) = c$.
        \item Determine $\text{ord} \alpha$ se $a_n = 2^n, \forall n \geq 0$.
    \end{enumerate}
\end{problem}

\begin{proof}
    \textbf{(a)} Note que, como $q_{n} = a_{n}q_{n-1} + q_{n-2} \geq 2q_{n-2}$, segue (por indução) que $q_n \geq C2^{n/2}$ para alguma constante positiva $C$.
    Portanto, para $1 < \gamma < \sqrt{2}$ e $n$ suficientemente grande, vale que $q_n \geq \gamma^n$  (na verdade, no pior caso,
    $q_n$ cresce como a sequência de Fibonacci). Portanto, se $\text{ord} \alpha > 2$, segue que
    \[
        \text{ord} \a - 2 = \limsup_{n\to\infty} \frac{\log{a_{n+1}}}{\log{q_n}} > \eps > 0
    \]
    para algum $\eps > 0$ e
    \[
        \limsup_{n\to\infty} \frac{\log{a_{n+1}}}{\log(\gamma^n)} \geq \limsup_{n\to\infty} \frac{\log{a_{n+1}}}{\log{q_n}} > \eps\\
    \]
    
    \[
        \limsup_{n\to\infty} \frac{\log{a_{n+1}}}{n\log\gamma} > \eps
    \]
    i.e, existe sequência infinita $(n_j)$ tal que 
    \[
        \frac{\log{a_{n_j+1}}}{n_j\log\gamma} > \eps \quad \text{e} \quad a_{n_j + 1} > (\gamma^{\eps})^{n_j}.
    \]
    Como $\gamma^{\eps} > 1$, tomando $1 < \beta < \gamma^{\eps}$, segue que para $n_j$ suficientemente grande,
    \[
        a_{n_j + 1} > (\gamma^{\eps})^{n_j} > \beta^{n_j + 1},
    \]
    finalizando a questão.

    \textbf{(b)} Suponha, por absurdo, que $\text{ord}\a < 1 + \exp(\limsup_{n\to\infty} \frac{\log\log(a_n + 1)}{n})$, então existem $\gamma, \gamma' \in \R$
    tal que
    \begin{equation}
        \label{l2:p3:eq1}
        2 + \limsup_{n\to\infty} \frac{\log a_{n+1}}{\log q_n} < \gamma' < \gamma < 1 + \exp(\limsup_{n\to\infty} \frac{\log\log(a_n + 1)}{n}).
    \end{equation}
    Da primeira desigualdade e da propriedade do $\limsup$ segue que existe $N \in \N$ tal que para todo $n \geq N$
    \[
        \frac{\log{a_{n+1}}}{\log q_n} < \gamma' - 2 \quad \text{e} \quad \log a_{n+1} < (\gamma' - 2)\log q_n,
    \]
    como $q_{n+1} = a_{n+1}q_n + q_{n-1} < 2a_{n+1}q_n$, segue que, para todo $n \geq N$
    \[
        \log q_{n+1} < \log 2 + \log{a_{n+1}} + \log q_n \leq \log(2) + (\gamma' - 1) \log q_n
    \]
    Resolvendo a recorrência até $N$, temos para $n > N$,
    \[
        \log q_{n} < \log 2 \cdot \bigg(\sum_{m = 0}^{n - N - 1} (\gamma' - 1)^m \bigg) + (\gamma' - 1)^{n - N} \log q_N
    \]
    ou seja, existe alguma constante positiva $K \in \R$, tal que, para todo $n$ maior que $N$,
    \[
        \log q_{n} < K(\gamma' - 1)^n.
    \]
    
    Pela segunda desigualdade de [\ref{l2:p3:eq1}], segue que existe sequência $(n_j)$ infinita, tal que para todo $j$ vale 
    \[
        \exp\bigg(\frac{\log \log (a_{n_j} + 1)}{n_j}\bigg) > \gamma - 1
    \]
    portanto
    \[
        \log (a_{n_j} + 1) > (\gamma - 1)^{n_j},
    \]
    pela concavidade do $\log$,
    \[
        \log a_{n_j} + 1 > \log a_{n_j} + \frac{1}{a_{n_j}} > \log(a_{n_j} + 1) > (\gamma - 1)^{n_j}
    \]
    e sendo bem gastosos e supondo $n_j$ suficientemente grande,
    \[
        \log a_{n_j} > \frac{(\gamma - 1)^{n_j}}{2}.
    \]
    Agora vamos calcular um lowerbound para $\text{ord} \alpha$ usando essa sequência, segue que
    \[
    \text{ord} \a > 2 + \lim_{j \to \infty} \frac{\log a_{n_j}}{\log q_{n_j - 1}} > 2 + \lim_{n_j \to \infty} \frac{(\gamma - 1)^{n_j}}{2K(\gamma'- 1)^{n_j - 1}} = \infty .
    \]
    o que é absurdo (na verdade provamos que deve ser que $\text{ord} \a = \infty$, de onde a desigualdade segue trivialmente).

    Para \textbf{(c)}, se $c = 2$ basta tomar $a_n = 1$ e o resultado segue. Caso $c > 2$, queremos $a_n \in \Z$ tal que
    \[
        1 + \exp(\lim_{n\to\infty} \frac{\log \log (a_n + 1)}{n}) = c
    \]
    desenvolvendo, uma alternativa seria
    \[
        a_n + 1 = \exp\exp(n\log(c-1)).
    \]
    Para facilitar as contas, vamos tomar $\alpha = [0; a_1, a_2, \dots]$ para $a_n$ sendo 
    \[
        a_n = \ceil{\exp\exp(n\log(c-1))} \in \N^*
    \]
    e veremos que $\text{ord} \alpha = c$. Já sabemos que 
    \begin{align*}    
        \text{ord}\alpha &\geq 1 + \exp(\limsup_{n\to\infty} \frac{\log\log (a_n + 1)}{n}) \\
        &\geq 1 + \exp(\limsup_{n \to \infty} \frac{\log\log(\exp\exp(n\log(c-1)))}{n})\\
        &\geq 1 + \exp(\limsup_{n \to \infty} \log(c-1)) = c
    \end{align*}
    Para a outra desigualdade, lembramos que
    \[
        q_{n} = a_{n}q_{n-1} + q_{n-2} \geq a_{n}q_{n-1}
    \]
    e portanto, por indução
    \[
        \log q_n \geq \sum_{j=1}^{n} \log a_j
    \]
    substituindo nossos valores e usando a soma da progressão aritmética, temos 
    \[
        \log q_n \geq \sum_{j=1}^{n} \exp(j\log(c-1)) = \frac{\exp((n+1)\log (c-1)) - \exp(\log (c-1))}{\exp(\log (c-1)) - 1} = \frac{(c-1)^{n+1} - (c-1)}{c - 2}.
    \]
    Temos também que (para $n$ suficientemente grande)
    \[
        \log a_{n+1} \leq \log(2\exp\exp((n+1)\log(c-1))) = \log(2) + \exp(n\log(c-1)) = \log(2) + (c-1)^{n+1}
    \]

    Logo,
    \[
        \text{ord} \alpha = 2 + \limsup_{n\to\infty} \frac{\log a_{n+1}}{q_n} \leq 2 + \limsup_{n\to\infty} \frac{(c-2)((c-1)^{n+1} + \log(2))}{(c-1)^{n+1} - (c-1)} = 2 + c - 2 = c.
    \]
    
    \textbf{(d)} Basta usar o que já conhecemos. De $q_{n} = a_n q_{n-1} + q_{n-2}$, temos 
    \[
        \sum_{i = 1}^{n} \log a_n \leq q_n \leq \sum_{i = 1}^{n} \log (a_n + 1) \leq \sum_{i = 1}^{n} \log (a_n + 1)  \leq \sum_{i = 1}^{n} \log(a_n) + \frac{1}{a_n}
    \]
    portanto, se $a_n = 2^n$, segue 
    \[
      \frac{n(n-1)\log 2}{2} = \log 2\bigg(\sum_{i=1}^{n} i \bigg) \leq \log q_n \leq \log 2\bigg(\sum_{i=1}^{n} i \bigg) + 1 = \frac{n(n-1)\log 2}{2} + 1
    \]
    ou seja $\log q_n = \Theta(n^2)$, substituindo na fórmula de $\text{ord} \alpha$, vemos 
    \[
        \text{ord} \alpha = 2 + \limsup_{n \to \infty} \frac{\log a_{n+1}}{\log q_n} = 2 + \limsup_{n\to\infty} \frac{(n+1)\log 2}{\Theta(n^2)} = 2.
    \]    
\end{proof}

\begin{problem}
    Não sei fazer esse, contas muito feias.
\end{problem}


\begin{problem}
    Prove que se $a$ e $b$ são inteiros positivos tais que $4ab - 1 \mid (4a^2 - 1)^2$, então 
    $a = b$.
\end{problem}

\begin{proof}
    Primeiramente, notamos que, sendo $(4ab - 1, b) = 1$, então
    \begin{align*}
        4ab - 1 \mid (4a^2 - 1)^2 &\iff 4ab - 1 \mid (4a^2 - 1)^2 \cdot b^2\\
        &\iff 4ab - 1 \mid (4a^2b - b)^2\\
        &\iff 4ab - 1 \mid (a - b)^2\\
        &\iff \exists k \in \N, (a-b)^2 = (4ab-1)k
    \end{align*}
    e portanto o problema é simétrico em relação as variáveis $a,b$. Suponha, sem perda de generalidade, que existe solução $(a,b)$ com $a > b$ e $a$ mínimo.
    Segue que existe $k \in \N$ tal que 
    \begin{align*}
        (a-b)^2 = (4ab-1)k \quad \text{ou seja} \quad a^2 - (2b + 4bk)a + b^2 + k = 0,
    \end{align*}
    note que $a$ é solução de $p(x) = x^2 - (2b + 4bk)x + b^2 + k = 0$. Sabemos que existe outra solução $a'$, tal que $p(a') = 0$ e 
    \begin{enumerate}[label=(\roman*)]
        \item $a'+a = 2b + 4bk$
        \item $a'\cdot a = b^2 + k$
    \end{enumerate}
    Agora, segue de (i) que $a'\in \Z$ e segue de (ii) que $a' = \frac{b^2 + k}{a} > 0$, logo $a'\in \N$. Como $p(a') = 0$, segue que $(a',b)$ é solução também.
    Sendo $k = (a-b)^2/(4ab - 1)$, segue que 
    \[ 
    a' = \frac{b^2 + \frac{(a-b)^2}{(4ab - 1)}}{a} < b + \frac{a^2}{3a^2} = b + \frac{1}{3}
    \]
    portanto, $a' \leq b$. Agora, se $a'= b$, então, por (i) $b \mid a$ e por (ii) $b \mid k$, logo $(a/b, 1)$ é uma soulução menor o que é absurdo.
    Se $a' < b$, temos que $(b, a')$ é uma solução menor e também temos absurdo.
\end{proof}

\begin{problem}
    Demonstre que $2x^2 - 219y^2 = -1$ não tem soluções inteiras, mas $2x^2 - 219y^2 \equiv -1 \pmod m$ tem soluções para todo inteiro positivo $m$.
\end{problem}

\begin{proof}
    Para a primeira parte, suponha que existe solução $(x_0,y_0) \in \N^2$ com $x_0$ mínimo. Seguindo a dica, notamos que 
    $(x_1,y_1) = (|293x_0 - 3066y_0|, |-28x_0 + 293y_0|)$, pois 
    \begin{align*}
        2(293x_0 - 3066y_0)^2 - 219(-28x_0 + 293y_0)^2=&\\
        [2(293)^2 - 219(28)^2] x_0^2 - [219(293)^2 - 2(3066)^2]y_0^2 &+ [219(2(28)(293)) - 2(2(293)(3066))]x_0y_0 =\\
        2x_0^2 - 219y_0^2 = -1
    \end{align*}
    (óbviamente). Agora, segue de $x_0$ ser minimal que $x_0 \leq x_1$ e portanto, 
    \[
        x_0 \leq 293x_0 - 3066y_0 \quad \lor \quad x_0 \leq 3066y_0 - 293x_0
    \]
    que é equivalente a
    \[
        x_0 \geq \frac{3066}{292}y_0 \quad \lor \quad x_0 \leq \frac{3066}{294}y_0
    \]
    mas verificamos que para $2x_0^2 - 219y_0^2 = -1$, devemos ter
    \[
        x_0^2 - \frac{219}{2}y_0^2 + \frac{1}{2} = 0
    \]
    ou seja, como estamos tomando $x_0$ positivo,
    \[
        \sqrt{\frac{218}{2}}y_0 < x_0 = \sqrt{\frac{219y_0^2 - 1}{2}} < \sqrt{\frac{219}{2}} y_0
    \]
    mas, usando uma calculadora, 
    \[
     (10.42...)y_0= \frac{3066}{294}y_0 < 10.4402y_0 < \sqrt{\frac{218}{2}}y_0 < 
     x_0 < \sqrt{\frac{219}{2}} y_0 < 10.4643 y_0 < \frac{3066}{292}y_0 = 10.5y_0
    \]
    absurdo.

    Para a segunda parte (mostrar que admite solução módulo $m$), vamos separar em casos. 
    
    Caso $m$ primo: se $m = 2$, então $(0,1)$ é solução; se $m = 3$, então 
    $(1,0)$ é solução; se $m = 73$, $(6,1)$ é solução. Se $m$ é qualquer outro primo, então $(m,2) = 1$ e $(m,219) = 1$ e portanto queremos 
    achar soluções módulo $m$ para 
    \[
        2x^2 \equiv 219y^2 - 1 \pmod m.
    \]
    Agora note que se definimos a função $f: \Z_m \to \Z_m$ dada por $f(x) = 2x^2 \pmod m$, segue, como há exatamente $(m-1)/2$ resíduos quadráticos, que $|f(\Z_m)| = (p+1)/2$.
    Da mesma forma, definindo $g : \Z_m \to \Z_m$ dada por $g(y) = 219y^2 - 1$, temos $|g(\Z_m)| = (p+1)/2$. Pelo principio da casa dos pombos, existe $(x,y) \in (\Z_m)^2$ tal que 
    $f(x) = g(y)$ e portanto solução $\pmod m$.

    Caso $m = p^\alpha$ para $p$ primo e $\alpha > 1$: seguindo a ideia do Joseph, vamos tentar provar por indução em $\alpha$.
    Seja $(x,y)$ solução para 
    \[
        2x^2 - 219y^2 + 1 \equiv 0 \pmod {p^{\alpha}} \quad \text{ou seja} \quad 2x^2 - 219y^2 + 1 = kp^{\alpha} 
    \]
    Se $k \equiv 0 \pmod p$, já temos o resultado. Podemos supor que $k \neq 0 \pmod p$. Vamos olhar para 
    as soluções $\pmod {p^\alpha}$ dadas por $(x + ap^\alpha, y + bp^\alpha)$ para $a,b \in \Z$, note que 
    \begin{align*}
        2(x + ap^\alpha)^2 - 219(y + bp^\alpha)^2 + 1 &\equiv (2x^2 - 219y^2 + 1) + (4xap^\alpha - 437ybp^\alpha) \pmod{p^{\alpha + 1}}\\
        & \equiv kp^\alpha + p^\alpha(4xa - 438yb)  \pmod{p^{\alpha + 1}}\\
        & \equiv p^\alpha(k + 4xa - 438yb)  \pmod{p^{\alpha + 1}}
    \end{align*}
    então basta que $k + 4xa - 438yb \equiv 0 \pmod p$. Se $(4x,p) = 1$, então independente do valor de $b$ e $k$, sempre conseguimos achar $a$ que satisfaz a equação.
    Se $(4x, p) = p$, ou $p = 2$ ou $p \mid x$. Se $p \mid x$, 
    \[
        219y^2 \equiv 2x^2 - 219y^2 \equiv -1 \pmod{p} 
    \]
    e portanto $(p,y) = 1$, donde segue que sempre podemos escolher $b$ a fim de satisfazer a equação. Caso $p = 2$ e $\alpha = 2$, então $(1,1)$ é solução, se $\alpha \geq 3$, buscamos soluções levemente diferentes.
    Olhemos para $(x+a2^{\alpha - 1}, y+b2^{\alpha-1})$ com $a,b \in \Z$. Repetindo a conta anterior, temos 
    \begin{align*}
        2(x+ a2^{\alpha - 1})^2 - 219(y + b2^{\alpha-1})^2 + 1 &\equiv (2x^2 - 219y^2 + 1) + ax2^{\alpha + 1} + a^22^{2\alpha - 1} - 219yb2^{\alpha} - 219b^22^{2\alpha - 2}\\
        &\equiv 2^\alpha(k - 219yb) \pmod{2^{\alpha + 1}}
    \end{align*}
    mas já estamos supondo $k$ ímpar e $y \equiv 1 \pmod 2$ pois $2x^2 - 219y^2 \equiv 1 \pmod 2$. Portanto $k - 219y \equiv 0 \pmod 2$ e tomando $b = 1$ temos solução.
    
    Agora moralmente terminamos, pois para $m = p^\alpha$ com $p$ primo sempre conseguimos solução e basta usar o Teorema Chinês dos Restos. Se
    $m = p_1^{\alpha_1}p_2^{\alpha_2}\dots p_n^{\alpha_n}$ com os $p_j$ primos distintos, então, 
    para cada $i$, encontramos solução
    \[
        2x_i^2 - 219y_i^2 \equiv -1 \pmod {p_i^{\alpha_i}}
    \]
    aplicando o TCM nos sistemas
    \[
        x \equiv x_i \pmod{p_i^{\alpha_i}} \quad \land \quad  y \equiv y_i \pmod{p_i^{\alpha_i}}  \quad \forall 1 \leq i \leq n
    \]
    encontramos $x \in \Z_m$ e $y \in \Z_m$ tal que 
    \[
        2x^2 - 219y^2 \equiv -1 \pmod m
    \]
    


    
\end{proof}