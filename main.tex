\documentclass{article}

\usepackage{amsmath,amssymb,amsthm,bbm,mathtools,calc,verbatim,enumitem,tikz,url,mathrsfs,cite,fullpage,hyperref,bm, marvosym}
\usepackage{dsfont}
\usepackage{float}
\usepackage{subcaption}
%\usepackage{setspace}
\renewcommand{\baselinestretch}{1.1}
\addtolength{\footskip}{\baselineskip/2}

%\usepackage{showlabels}
\usepackage{comment}
\usepackage[english]{babel}

%No caso do livro o Tu, para decoplar seções de capitulos usamos:
% \usepackage{chngcntr}
% \counterwithout{section}{chapter}

\theoremstyle{definition}
\newtheorem{theorem}{Theorem}[section]
\newtheorem{lemma}[theorem]{Lemma}
\newtheorem{corollary}[theorem]{Corollary}
\newtheorem{prop}[theorem]{Proposition}
\newtheorem{observation}[theorem]{Observation}
\newtheorem{construction}[theorem]{Construction}

\newtheorem{definition}[theorem]{Definition}

\newtheorem{conjecture}[theorem]{Conjecture}
\newtheorem{question}[theorem]{Question}
\newtheorem{obs}[theorem]{Observation}
\newtheorem{claim}[theorem]{Claim}
\newtheorem{fact}[theorem]{Fact}
\newtheorem{problem}{Problem}[section]
\newtheorem{exercise}{Exercise}[section]
\newtheorem{remark}[theorem]{Remark}

% my custom problems
\newtheorem{innercustomexercise}{Exercise}
\newenvironment{customexercise}[1]
  {\renewcommand\theinnercustomexercise{#1}\innercustomexercise}
  {\endinnercustomexercise}

\newenvironment{clmproof}[1]{\begin{proof}[Proof of Claim~\ref{#1}]\let\qednow\qedsymbol\renewcommand{\qedsymbol}{}}{\; \qednow \end{proof}}

\newcounter{step}
\newenvironment{proofsteps}{
  \setcounter{step}{0}
  \begin{list}{\textbf{Step \arabic{step}.}}{\usecounter{step}
  \setlength{\leftmargin}{1.5em}
  \setlength{\itemsep}{0.3em}}
}{
  \end{list}
}

\newcommand{\stepcomment}[1]{\hfill {\small\textit{#1}}}


\newcommand\N{\mathbb{N}}
\newcommand\R{\mathbb{R}}
\newcommand\Z{\mathbb{Z}}
\newcommand\cA{\mathcal{A}}
\newcommand\cB{\mathcal{B}}
\newcommand\cN{\mathcal{N}}
\newcommand\cP{\mathcal{P}}
\newcommand\cQ{\mathcal{Q}}
\newcommand\cZ{\mathcal{Z}}
\newcommand\rN{\tilde{N}}
\newcommand\cT{\mathcal{T}}
\newcommand\cE{\mathcal{E}}
\def\Pr{\mathbb{P}}
\def\cS{\mathcal{S}}
\newcommand\Ex{\mathbb{E}}
\newcommand\id{\hbox{$1\mkern-6.5mu1$}}
\newcommand\lcm{\operatorname{lcm}}
\newcommand\eps{\varepsilon}
\newcommand{\floor}[1]{\lfloor #1 \rfloor}
\newcommand{\ceil}[1]{\lceil #1 \rceil}
\newcommand{\prob}{\begin{problem} \end{problem}}
\newcommand{\exer}{\begin{exercise} \end{exercise}}
\newcommand{\cexer}[1]{\begin{customprob}{#1}\end{customprob}}


\renewcommand{\leq}{\leqslant}
\renewcommand{\geq}{\geqslant}
\renewcommand{\le}{\leqslant}
\renewcommand{\ge}{\geqslant}
\renewcommand{\to}{\rightarrow}
\renewcommand{\Re}{\re}

\def\ds{\displaystyle}

\def\eps{\varepsilon}
\def\p{\partial}

\def\HH{\mathcal{H}}
\def\E{\mathbb{E}}
\def\C{\mathbb{C}}
\def\cM{\mathcal{M}}
\def\cF{\mathcal{F}}
\def\cI{\mathcal{I}}
\def\R{\mathbb{R}}
\def\bS{\mathbb{S}}
\def\bH{\mathbb{H}}
\def\Z{\mathbb{Z}}
\def\N{\mathbb{N}}
\def\PP{\mathbb{P}}
\def\1{\mathbbm{1}}
\def\l{}
\def\k{\kappa}
\def\w{\omega}
\def\s{\sigma}
\def\t{\theta}
\def\a{\alpha}
\def\g{\gamma}
\def\z{\zeta}
\def\zbar{\bar{z}}
\def\<{\langle}
\def\>{\rangle}
%\def\endproof{{\hfill $\square$} }
\def\Xt{\widetilde{X}}
\def\Pt{\widetilde{P}}

\def\cN{\mathcal{N}}
\def\cC{\mathcal{C}}
\def\cD{\mathcal{D}}
\def\cR{\mathcal{R}}
\def\cB{\mathcal{B}}
\def\cG{\mathcal{G}}
\def\EE{\mathbb{E}}
\def\FF{\mathbb{F}}
\def\T{\mathbb{T}}
\def\cA{\mathcal{A}}
\def\cQ{\mathcal{Q}}
\def\cC{\mathcal{C}}
\def\F{\mathbb{F}}
\def\tm{\tilde{\mu}}
\def\ts{\tilde{\sigma}}
\def\Q{\mathcal{Q}}
\def\vp{\varphi}

\hypersetup{
	colorlinks=true,
	linkcolor=blue,
	urlcolor=blue,
}

\pagestyle{plain}
\author{
	Aluno: Henrique Lima Cardoso\\
	}
\title{Listas de Teoria dos Números}

\begin{document}
\maketitle

\tableofcontents
\setcounter{section}{-1}

\section{Introdução}
Ao decorrer do curso, vou escrever minhas resoluções dos exercícios nesse arquivo. Tem alguns motivos para isso:
\begin{enumerate}
	\item Posso reutilizar resultados passados.
	\item Está tudo organizado se um futuro henrique quiser rever.
\end{enumerate}
O código fonte pode ser encontrado em \url{https://github.com/hnrq104/tn}.

\subsection{Notação}
Até agora encontrei os seguintes usos de notação não convencional:
\begin{itemize}
  \item $x \equiv \{a_1, a_2, \dots\} \pmod c$  - significa que $x$ é congruente a um elemento do conjunto $\{a_i\}$.
  \item $x \equiv \frac{a}{b} \pmod m$ - significa que $x \equiv ab^{-1} \pmod m$.
  \item Se $\sum_{n \geq 1} a_n$ é uma série de termos positivos, denoto $\sum_{n \geq 1} a_n < \infty$ se a série converge.
\end{itemize}
%LISTA 1
\pagebreak
\section{Lista 1 - \date{12/1/2026}}

\begin{problem}
    Dados inteiros positivos $a, b$ e $c$, dois a dois primos entre si, demonstre que $2abc - ab - bc -ca$ é o 
    maior número inteiro que não pode expressar-se na forma $xbc + yca + zab$ com $x,y$ e $z$ inteiros não negativos.
\end{problem}

\begin{proof}
    Note que como $(b,c) = 1$, temos que $(ab,ac) = a$ e, portanto por Bachét-Bezout existe solução
    para $z'ab + y'ca = a$ com $z',y'$ inteiros. Por sua vez, como $(a,bc) = 1$, existe solução para 
    $ma + nbc = 1$ com $m,n$ inteiros. Juntando as duas equações, encontramos $mz'ab + my'ca + nbc = 1$ que é solução para 
    a equação $xbc + yca + zab = 1$ e, portanto, temos soluções para $xbc + yca + zab = k$ para qualquer inteiro $k$.

    Vamos mostrar que $2abc - ab - bc -ca$ não pode ser escrito como $xbc + yca + zab$ para $x,y,z \in \N$. Suponha, que conseguimos,
    temos 
    \begin{align*}
        2abc - ab - bc -ca = xbc + yca + zab\\
        2abc = (x+1)bc + (y+1)ca + (z+1)ab 
    \end{align*}
    tomando a segunda equação módulo $a$, achamos
    \[
        0 \equiv (x+1)bc \pmod a \Rightarrow x+1 \equiv 0 \pmod a
    \]
    ou seja, $a \mid (x+1)$. Como $x \geq 0$, devemos ter $(x+1) \geq a$. Simetricamente (tomando módulo $b$ e depois $c$), sabemos que $(y+1) \geq b$ e $(z+1) \geq c$.
    Mas já encontramos contradição, uma vez que essas desigualdades implicam
    \[
    (x+1)bc + (y+1)ca +(z+1)ab \geq abc + bca + cab = 3abc > 2abc
    \]

    Agora seja $n > 2abc - bc - ac - ab$, mostraremos que existe solução natural para $n = xbc + yac + zab$. Primeiro, vamos
    caracterizar as soluções inteiras, que existem pela observação anterior. Note que se $(x,y,z)$ e $(x',y',z')$ são soluções,
    então
    \begin{equation}
        \label{lista1:eq:prob1}
        (x-x')bc + (y-y')ac + (z-z')ab = 0
    \end{equation}
    tomando a equação módulo $a$, vemos que $(x-x') \equiv 0 \pmod a$ e portanto $x' = x + ra$ para algum $r \in \Z$. Simetricamente,
    vemos que $y' = y + sb$ e $z' = z + tc$ para $s,t \in \Z$. Portanto, [\ref{lista1:eq:prob1}] se expressa como 
    \[
        (ra)bc + (sb)ac + (tc)ab = (r+s+t)abc = 0 \iff (r+s+t) = 0 
    \]
    Ou seja, se $(x_0, y_0, z_0)$ é uma solução inicial, todas as outras soluções são da forma $(x_0 + ra, y_0 + sb, z_0 + tc)$ onde 
    $r + s + t = 0$, é fácil ver que qualquer qualquer tripla dessa forma também satisfaz a equação original. Nosso problema se resume então
    a encontrar soluções inteiras $(r,s,t)$ para a seguinte série de relações:
    \begin{align*}
        x_0 + ra > -1\\
        y_0 + sb > -1\\
        z_0 + tc > -1\\
        r + s + t = 0
    \end{align*}
    Isolando as variáveis e escrevendo $t$ como $-(r+s)$, temos
    \begin{align*}
        -\frac{(x_0+1)}{a} < r\\
        -\frac{(y_0+1)}{b} < s\\
        r+s < \frac{(z_0 + 1)}{c}
    \end{align*}
    As duas primeiras desigualdades, implicam que 
    \[
       - \bigg(\frac{(x_0+1)}{a} + \frac{(y_0+1)}{b}\bigg) < r+s < \frac{(z_0 + 1)}{c}
    \]
    Notamos (seguindo a resolução do livro para um problema similar) que 
    \[
        \frac{(z_0 + 1)}{c} - - \bigg(\frac{(x_0+1)}{a} + \frac{(y_0+1)}{b}\bigg) = \frac{(z_0 + 1)}{c} + \frac{(x_0+1)}{a} + \frac{(y_0+1)}{b} =
        \frac{n + bc + ac + ab}{abc} > 2
    \]
    pois $n > 2abc - bc - ac - ab$. Segue que o intervalo $\bigg(- \frac{(x_0+1)}{a} - \frac{(y_0+1)}{b}, \frac{(z_0 + 1)}{c}\bigg)$ tem ao menos dois inteiros.
    Particularmente, os números %mais casos aqui
    \[ 
        \bigg\lceil - \frac{(x_0+1)}{a} - \frac{(y_0+1)}{b} \bigg\rceil \quad \text{e} \quad \bigg\lceil - \frac{(x_0+1)}{a} - \frac{(y_0+1)}{b} \bigg\rceil + 1
    \]
    pertencem ao intervalo. Tomando
    \begin{align*}
        r = \begin{cases}
            \ceil{-(x_0+1)/a} & \text{se } -(x_0+1)/a \not\in \Z\\
            \ceil{-(x_0+1)/a} + 1 & \text{se } -(x_0+1)/a \in \Z
        \end{cases}
    \end{align*}
    e $s$ análoga, sendo 
    \begin{align*}
        s = \begin{cases}
            \ceil{-(y_0+1)/b} & \text{se } -(y_0+1)/b \not\in \Z\\
            \ceil{-(y_0+1)/b} + 1 & \text{se } -(y_0+1)/b \in \Z
        \end{cases}
    \end{align*}
    achamos soluções $(r,s,t)$ compatíveis com o sistema de desigualdades.

\end{proof}

\begin{problem}
    Seja $p$ um número primo ímpar. Seja $s$ o menor inteiro positivo que não é resíduo quadrático módulo $p$.
    \begin{enumerate}[label=(\alph*)]
        \item Mostre que $p > s^2 - s$.
        \item Suponha que $p > 5$ e que $-1$ seja resíduo quadrático módulo $p$: mostre que $p>2s^2 - s$.
    \end{enumerate}
\end{problem}


\begin{proof}
    
\end{proof}


\begin{problem}
    Seja $p$ um primo ímpar, $a$ um inteiro e $n$ um inteiro positivo. Sejam $\alpha$ e $\beta$ inteiros negativos, com $\alpha > 0$. Prove:
    \begin{enumerate}[label=(\alph*)]
        \item Se $p^\beta$ e $p^\alpha$ são as maiores potências de $p$ que dividem $n$ e $a-1$ respectivamente então $p^{\alpha + \beta}$ é 
        a maior potência que divide $a^n - 1$.
        \item Se $n$ é ímpar e $p^\beta$ e $p^\alpha$ são as maiores potências de $p$ que dividem $n$ e $a+1$ respectivamente então $p^{\alpha + \beta}$ é a maior potência
        de $p$ que divide $a^n + 1$.
    \end{enumerate}
\end{problem}


\begin{proof}
    
\end{proof}


\begin{problem}
    \begin{enumerate}[label=(\alph*)]
        \item Prove que $\text{ord}_{2^k} 5 = 2^{k-2}$, para todo $k \geq 2$.
        \item Prove que se $a$ é um inteiro ímpar e $k \geq 2$ então existem $\eps_j \in \{-1,1\}$ e $j \in \mathbb{Z}$ com $0 \leq j \leq 2^{k-2}$, únicamente determinados,
        tais que $a \equiv \eps_j \cdot 5^j \mod 2^k$.
    \end{enumerate}
\end{problem}

\begin{proof}
    
\end{proof}

\begin{problem}
    Qual é o menor natural $n$ para o qual existe $k$ natural de modo que os 2026 últimos dígitos na representação decimal de $n^k$ são iguais a $1$?
\end{problem}


\begin{proof}
    
\end{proof}

\begin{problem}
    O símbolo de Legendre $\big(\frac{a}{p}\big)$ pode ser estendido para o símbolo de Jacobi $\big(\frac{a}{n}\big)$, que está definido para $a$ inteiro 
    arbitrário e $n$ inteiro positivo ímpar por $\big(\frac{a}{n}\big) = \big(\frac{a}{p_1}\big)^{\alpha_1} \dots \big(\frac{a}{p_k}\big)^{\alpha_k}$ se 
    $n = p_1^{\alpha_1}\dots p_k^{\alpha_k}$ é a fatoração prima de $n$ (onde os $\big(\frac{a}{p_j}\big)$ são dados pelo símbolo de Legendre usual); temos 
    $\big(\frac{a}{1}\big) = 1$ para todo inteiro $a$.

    Prove as seguintes propriedades do símbolo de Jacobi, que podem ser usadas para calcular rapidametne símbolos de Legendre (e de Jacobi):
    \begin{enumerate}
        \item Se $a \equiv b \mod n$ então $\big(\frac{a}{n}\big) = \big(\frac{b}{n}\big)$.
        \item $\big(\frac{a}{n}\big) = 0$ se $\gcd(a,n) \neq 1$ e $\big(\frac{a}{p}\big) \in \{-1,1\}$ se $\gcd(a,n) = 1$.
        \item $\big(\frac{ab}{n}\big) = \big(\frac{a}{n}\big)\big(\frac{b}{n}\big)$; em particular, $\big(\frac{a^2}{n}\big) \in \{0,1\}$.
        \item $\big(\frac{a}{mn}\big) = \big(\frac{a}{n}\big)\big(\frac{a}{m}\big)$; em particular, $\big(\frac{a}{n^2}\big) \in \{0,1\}$.
        \item Se $m$ e $n$ são positivos e ímpares, então $\big(\frac{m}{n}\big) = (-1)^{(m-1)/2 \cdot (n-1)/2} \big(\frac{n}{m}\big)$.
        \item $\big(\frac{-1}{n}\big) = (-1)^{(n-1)/2}$.
        \item $\big(\frac{2}{n}\big) = (-1)^{(n^2 - 1)/8}$ se $n$ é ímpar.
    \end{enumerate}
\end{problem}


\begin{proof}
\textbf{(1)} Note que se $a \equiv b \pmod n$, então $a \equiv b \pmod{p_j}$ para todo $1 \leq j \leq k$. Pela propriedade usual do símbolo de Legendre,
$\big( \frac{a}{p_j} \big) = \big( \frac{b}{p_j}\big)$ para todo $j$ e, portanto,
\[
    \bigg(\frac{a}{n}\bigg) = \prod_{j=1}^{r}\bigg( \frac{a}{p_j} \bigg)^{\alpha_j} = \prod_{j=1}^{r}\bigg( \frac{b}{p_j} \bigg)^{\alpha_j} = \bigg(\frac{b}{n}\bigg).
\]

\textbf{(2)} Se $(a,n) \neq 1$, então existe algum primo $p_i$ tal que $p_i \mid a$, portanto $\big(\frac{a}{p_i}\big) = 0$ e 
\[ \prod_{j=1}^{r}\bigg( \frac{a}{p_j} \bigg)^{\alpha_j} = 0.\]
Por outro lado, se $(a,n) = 1$, então para todos os primos $p_i$, temos que $p_i \nmid a$ e temos $\big(\frac{a}{p_i}\big) \in \{-1,1\}$. Portanto
\[
    \prod_{j=1}^{r}\bigg( \frac{a}{p_j} \bigg)^{\alpha_j} \in \{-1,1\}.
\]

\textbf{(3)}
Basta abrir a conta e usar a propriedade dos símbolos usuais de Legendre,
\[
\bigg(\frac{ab}{n}\bigg) = \prod_{j=1}^{r}\bigg( \frac{ab}{p_j} \bigg)^{\alpha_j} = \prod_{j=1}^{r}\bigg( \frac{a}{p_j} \bigg)^{\alpha_j}\bigg( \frac{b}{p_j} \bigg)^{\alpha_j} =
\prod_{j=1}^{r}\bigg( \frac{a}{p_j} \bigg)^{\alpha_j} \prod_{j=1}^{r} \bigg( \frac{b}{p_j} \bigg)^{\alpha_j} =  
\bigg(\frac{a}{n}\bigg)\bigg(\frac{b}{n}\bigg).
\]

\textbf{(4)} Sejam $q_1 \dots q_r$ os primos que dividem $n$ ou $m$. Escrevemos $n = q_1^{\alpha_1}\dots q_r^{\alpha_r}$ e $m = q_1^{\beta_1} \dots q_r^{\beta_r}$
onde os $\alpha_i$ e $\beta_j$ podem potencialmente ser $0$. Temos
$$nm = q_1^{\alpha_1 + \beta_1}\dots q_r^{\alpha_r + \beta_r}$$
e logo
$$\bigg( \frac{a}{nm} \bigg) = \prod_{j = 1}^{r} \bigg(\frac{a}{q_j}\bigg)^{\alpha_j + \beta_j} = \prod_{j = 1}^{r} \bigg(\frac{a}{q_j}\bigg)^{\alpha_j} \prod_{j = 1}^{r} \bigg(\frac{a}{q_j}\bigg)^{\beta_j}.$$
Agora notamos que se $q_j \nmid n$, então $\alpha_j = 0$ e  se $q_i \nmid m$, então $\beta_i = 0$, então os produtórios acima se expressam como
\[
    \bigg( \frac{a}{nm} \bigg) =  \prod_{q_j \mid n} \bigg(\frac{a}{q_j}\bigg)^{\alpha_j} \prod_{q_i \mid m} \bigg(\frac{a}{q_i}\bigg)^{\beta_i} = \bigg(\frac{a}{n}\bigg) \bigg(\frac{a}{m}\bigg).
\]

\textbf{(5)} Esse é mais interessante, vamos usar reciprocidade quadrática e as propriedades anteriores. Primeiramente, note que por \textbf{(2)}, a fórmula é válida se $(m,n) \neq 1$,
já que tanto $\big(\frac{m}{n}\big) = 0$ quanto $\big(\frac{n}{m}\big) = 0$. Podemos supor então que $(m,n) = 1$. Outro caso de interesse é que se $a^2 \mid m$, então $m = a^2m'$ e por \textbf{(3)},
$\big(\frac{m}{n}\big) = \big(\frac{m'}{n}\big)$. Já que o mesmo vale para o "denominador" do símbolo de Legendre, podemos supor ainda mais que $m$ e $n$ são livres de quadrados. Ou seja, podemos considerar (ad hoc) que suas fatorações 
são $n = p_1\dots p_l \cdot r_1 \dots r_k $ e $m = q_1 \dots q_t \cdot s_1 \dots s_h$ onde os $p_i \equiv q_j \equiv 1 \pmod 4$, os $r_i \equiv s_j \equiv 3 \pmod 4$ e os primos das fatorações são todos distintos.

Após todas nossas suposições, temos (usando a propriedade \textbf{(3)} e \textbf{(4)}  várias vezes)
\[
    \bigg(\frac{m}{n}\bigg) = \bigg(\frac{q_1 \dots q_t \cdot s_1 \dots s_h}{p_1\dots p_l \cdot r_1 \dots r_k}\bigg)
    = \bigg(\frac{q_1 \dots q_t}{p_1\dots p_l}\bigg)\bigg(\frac{s_1 \dots s_h}{p_1\dots p_l}\bigg) \bigg(\frac{q_1 \dots q_t}{r_1 \dots r_k}\bigg)\bigg(\frac{s_1 \dots s_h}{r_1 \dots r_k}\bigg),
\]
ou seja,
\[
    \bigg(\frac{m}{n}\bigg) = \prod_{(q_i,p_j)} \bigg(\frac{q_i}{p_j}\bigg) \cdot \prod_{(s_i,p_j)} \bigg(\frac{s_i}{p_j}\bigg) \cdot \prod_{(q_i,r_j)} \bigg(\frac{q_i}{r_j}\bigg) \cdot \prod_{(s_i,r_j)} \bigg(\frac{s_i}{r_j}\bigg).
\]
Pela lei da reciprocidade quadrática, se $h$ é um primo com $h \equiv 1 \pmod 4$ e $g$ é outro primo qualquer, então $\big(\frac{h}{g}\big) = \big(\frac{g}{h}\big)$ e se ambos $g$ e $h$ forem congruentes a 3 módulo 4, então 
$\big(\frac{h}{g}\big) = -\big(\frac{g}{h}\big)$. Podemos usar isso na expressão acima para obter
\[
    \bigg(\frac{m}{n}\bigg) = \prod_{(q_i,p_j)} \bigg(\frac{p_j}{q_i}\bigg) \cdot \prod_{(s_i,p_j)} \bigg(\frac{p_j}{s_i}\bigg) \cdot \prod_{(q_i,r_j)} \bigg(\frac{r_j}{q_i}\bigg) \prod_{(s_i,r_j)} - \bigg(\frac{r_j}{s_i}\bigg),
\]
de forma que (juntando os produtórios)
\[
    \bigg(\frac{m}{n}\bigg) = (-1)^{kh} \bigg(\frac{n}{m}\bigg).
\]
Para o resultado, basta mostrar que $kh \equiv (n-1)/2 \cdot (m-1)/2 \pmod 2$ (note que são inteiros uma vez que $n$ e $m$ são ímpares). Vamos olhar para $n$ e $m$ módulo 4, observamos que 
\[
    n \equiv p_1\dots p_l \cdot r_1 \dots r_k \equiv r_1 \dots r_k \equiv 3^k \equiv \begin{cases}
        1 &\text{se } k \equiv 0 \mod 2\\
        3 &\text{se } k \equiv 1 \mod 2
    \end{cases}\bigg\} \pmod 4
\]
o resultado análogo segue para $m$ e $h$. Disso já obtemos que se $h$ ou $k$ forem pares, então $n$ ou $m$ são 1 módulo 4, portanto $(n-1)/2$ ou $(m-1)/2$ é par
e $hk \equiv 0 \equiv (n-1)/2 \cdot (m-1)/2 \pmod 2$. Se ambos $h$ e $k$ forem ímpares, então $n \equiv m \equiv 3 \pmod 4$, logo $(n-1)/2$ e $(m-1)/2$ são ímpares
e $hk \equiv 1 \equiv (n-1)/2 \cdot (m-1)/2 \pmod 2$ concluindo a demonstração.

\textbf{(6)} Vamos fazer uma análise semelhante a \textbf{(5)}. Pelo observado anteriormente, podemos supor que $n$ é livre de quadrados e se escreve
$n = p_1 \dots p_l \cdot r_1 \dots r_k$ com os $p_i \equiv 1 \pmod 4$ e $r_i \equiv 3 \pmod 4$. Abrindo o símbolo de Jacobi, temos então
\[
    \bigg(\frac{-1}{n}\bigg) = \prod_{p_i} \bigg(\frac{-1}{p_i}\bigg) \prod_{r_j} \bigg(\frac{-1}{r_j}\bigg).
\]
Como $\big(\frac{-1}{x}\big) = 1$ se $x$ é primo e $x \equiv 1 \pmod 4$ e $\big(\frac{-1}{x}\big) = -1$ se $x$ for um primo com $x \equiv 3 \pmod 4$, segue que 
\[
    \bigg(\frac{-1}{n}\bigg) = (-1)^{k}
\] 
ou seja, para mostrar a igualdade, basta veriricar que $k \equiv (n-1)/2 \pmod 2$ e já fizemos isso na prova da propriedade anterior.

\textbf{(7)} Seguindo a mesma ideia, vamos fatorar $n$ de maneira esperta. Vimos que, sem perda de generalidade, podemos supor $n$ livre de quadrados, então escrevemos a fatoração prima de $n$ como
\[
    n = (p_1^+p_2^+\dots p_l^+) \cdot (p_1^-p_2^-\dots p_k^-) \cdot (q_1^+q_2^+\dots q_r^+) \cdot (q_1^-q_2^-\dots q_s^-)
\]
onde cada $p_i^+ \equiv 1 \pmod 8$, $p_i^- \equiv -1 \pmod 8$, $q_i^+ \equiv 3 \pmod 8$ e $q_i^- \equiv -3 \pmod 8$. Usando a propriedade \textbf{(4)}, temos 
\[
    \bigg(\frac{2}{n}\bigg) = \prod_{p_i^+} \bigg(\frac{2}{p_i^+}\bigg) \cdot \prod_{p_i^-} \bigg(\frac{2}{p_i^-}\bigg) \cdot \prod_{q_i^+} \bigg(\frac{2}{q_i^+}\bigg) \cdot \prod_{q_i^-} \bigg(\frac{2}{q_i^-}\bigg).
\]
Por reciprocidade quadrática, sabemos que para todo $i$ vale $\big(\frac{2}{p_i^+}\big) = \big(\frac{2}{p_i^-}\big) = 1$ e $\big(\frac{2}{q_i^+}\big) = \big(\frac{2}{q_i^-}\big) = -1$, portanto, a equação acima reduz-se para
\[
    \bigg(\frac{2}{n}\bigg) = (-1)^{r + s}.
\]
Para finalizar a demonstração, basta mostrar que $r+s \equiv (n^2-1)/8 \pmod 2$ ou, equivalentemente, desejamos mostrar 
\[
    r + s \equiv 0 \pmod 2 \iff n \equiv \{-1,1\} \pmod 8 \quad \text{e} \quad r + s \equiv 1 \pmod 2 \iff n \equiv \{-3,3\} \pmod 8.
\]
Notamos primeiramente que
\[
    n = (p_1^+p_2^+\dots p_l^+) \cdot (p_1^-p_2^-\dots p_k^-) \cdot (q_1^+q_2^+\dots q_r^+) \cdot (q_1^-q_2^-\dots q_s^-) \equiv (1)^l \cdot (-1)^k \cdot (3)^r \cdot (-3)^s \pmod 8,
\]
ou seja, $n \equiv \eps  \cdot (3)^r \cdot (-3)^s \pmod 8$ onde $\eps \in \{-1,1\}$. Agora para a análise de casos. Se $r + s$ for par, então ou $r$ e $s$ são pares 
onde $n \equiv \eps  \cdot 1 \cdot 1 \in \{-1,1\} \pmod 8$ ou $r$ e $s$ são ímpares e temos $n \equiv \eps  \cdot 3 \cdot -3 \equiv -\eps \in \{-1,1\} \pmod 8$. Se, por outro lado,
$r + s$ for ímpar, então ou $r$ é ímpar e $s$ é par onde $n \equiv \eps \cdot 3 \cdot 1 \in \{3,-3\} \pmod 8$ ou $r$ é par e $s$ é ímpar, onde também temos $n \equiv \eps \cdot 1 \cdot -3 \in \{3,-3\} \pmod 8$. O que 
conclui a demonstração.



\end{proof}


%LISTA 2
\pagebreak
\section{Lista 2 - 26/01/2026}

\begin{problem}
    Sejam $a,n \in \N^*$ e considere a sequência $(x_k)$ definida por $x_1 = a, x_{k+1} = a^{x_k}$ para todo $k \in \N$. Demonstre que existe $N \in \N$ tal que 
    $x_{k+1} \equiv x_k \pmod n$ para todo $k \geq N$.
\end{problem}

\begin{obs}
Note que, se $a > 1$, a sequência $(x_k)$ é estritamente crescente já que $x_{k+1} = a^{x_k} \geq 2^{x_k} > x_k$ e tende para infinito.
\end{obs}

\begin{proof}
    Note que se $a = 1$, então o resultado segue pois $x_k = 1$ para todo $k$.
    Vamos provar por indução em $n$ para $a \neq 1$. Para $n = 1$ é óbvio, uma vez que $x_{k} \equiv 0 \pmod 1$ para todo $k$ independente de $a$.
    Seja $n > 1$ e suponha que vale para todo $m < n$, i.e. existe $N_m$ tal que se $k \geq N_m$, então $x_{k+1} \equiv x_k \pmod m$.
    Fatorando $n$, podemos supor que
    \[
        n = \prod_{i=1}^{r} p_i^{\alpha_i}
    \]
    com os $p_i$ sendo primos distintos. Pelo Teorema Chinês dos Restos, o problema se resume a mostrar que existe $N$ tal que para todo $k \geq N$,
    \[
        \forall 1 \leq i \leq r,\quad x_{k+1} \equiv x_{k} \pmod{p_i^{\alpha_i}}.
    \]
    Vamos encontrar um $N_i$ para cada $p_i$ e tomar $N = \max \,\{N_i : 1 \leq i \leq r\}$.
    Se $(a,p_i) = p_i$, temos $a = p_iq$ e basta que $x_k \geq \alpha_i$ para $x_{k+1} = a^{x_k} = (p_i\cdot q)^{x_k} \equiv 0 \pmod{p_i^{\alpha_i}}$. Como $x_k \to \infty$,
    existe $N_i$ satisfazendo $x_k \equiv 0 \pmod{p_i^{\alpha_i}}$ para todo $k \geq N_i$. Por outro lado, se $(a,p_j) = 1$, podemos
    usar indutivamente o resultado para $m = \varphi(p_j^{\alpha_j}) < n$ de onde segue que existe $M_j$ tal que
    \[
        x_{k+1} \equiv x_{k} \pmod{m} \quad \forall k \geq M_j,
    \]
    portanto
    \[
        x_{k+2} = a^{x_{k+1}} = a^{m\cdot t + x_{k}} \equiv a^{x_k} = x_{k+1} \pmod {p_j^{\alpha_j}} \quad \forall k \geq M_j
    \]
    e portanto, podemos tomar $N_j = M_j + 1$. Tomando $N$ o máximo dos $N_i$, segue que 
    \[
    \forall k \geq N, \forall 1 \leq i \leq r \quad x_{k+1} \equiv x_{k} \pmod{p_i^{\alpha_i}}
    \]
    e, por fim,
    \[
    x_{k+1} \equiv x_{k} \pmod{n} \quad \forall k \geq N
    \]    

\end{proof}

\begin{problem}
    
\end{problem}

%LISTA 3
\pagebreak
\section{Lista 3 - 09/02/2026}

\begin{problem}
    Uma pulseira é formada por pedras coloridas, de mesmo tamanho,
    pregadas em volta de um círculo de modo a ficarem igualmente espaçadas.
    Duas pulseiras são consideradas iguais se, e somente se, suas configurações
    de pedras coincidem por uma rotação. Se há pedras disponíveis de $k \geq 1$ cores 
    distintas, mostre que o número de pulseiras diferentes possíveis 
    com $n$ pedras é dado pela expressão
    \[
    \frac{1}{n} \sum_{d\mid n} \varphi(d) \cdot k^{n/d}
    \]
\end{problem}

Precisaremos de uma ferramenta famosa para contagens desse tipo.
\begin{lemma}
    (Burnside) Seja $G \curvearrowright X$ uma ação de um grupo finito $G$ sob um conjunto finito $X$. Dado $g \in G$,
    denotamos $N(g) = \{ x \in X\;|\; g\cdot x = x\}$, i.e. os elementos fixados por $g$. Então vale a seguinte igualdade
    \[
        |X/G| = \frac{1}{|G|} \sum_{g \in G} |N(g)|.
    \]
\end{lemma}
A prova desse lema envolve somente contagem e o teorema de Lagrange. Se necessário, provarei, ele em um apêndice.
Agora podemos dar seguimento ao problema.

\begin{proof}
    (do Exercício) Conseguimos encaixar precisamente o contexto dado no problema nas condições do Lema. Definimos $X = [k]^n$
    os vetores com $n$ entradas e coordenadas em $\{1,2 \dots k\}$ e seja $G$ o grupo das rotações das pulseiras, podemos tomar $G \cong (\Z_n, +)$.
    Dada $r \in G$ rotação e $\vec{x} = (x[1], x[2], \dots x[n]) \in X$ vetor de pedras coloridas, definimos a ação
    \[
        (r\cdot \vec{x})[(i+r)] = x[i] 
    \]
    onde os índices são tomados módulo $n$. Por exemplo, se $n = 3$, $k = 3$, $\vec{x} = (1,2,3)$ e $r = \bar{2}$,
    temos $r\cdot \vec{x} = (2,3,1)$, rodamos as entradas do vetor em duas posições para a direita.

    Notamos que as classes em $|X/G|$ são exatamente as pulseiras em que estamos interessados em contar. Aplicando o lema, teriamos
    \[
        \#\{\text{pulseiras}\} = \frac{1}{n} \sum_{r \in \Z_n} \#\{\text{vetores fixados por  } r \}
    \]
    Então basta calcular quantos vetores são fixados por cada rotação $r \in Z_n$. Se $r \cdot x = x$, segue que para todo $i \in \Z_n$, $x[i+r] = x[i]$,
    consequentemente, para todo $a \in \Z$ e portanto para todo $a \in \Z_n$
    \[
        x[i + ar] = x[i]
    \]
    onde novamente aqui as somas nos índices são tomadas módulo $n$. Isso é, se fixarmos a cor de $x[i]$, fixamos a cor de todos os índices
    $\{i + ar \pmod n \;|\; a \in \Z_n\}$. Mas, se $d = (r,n)$, 
    \[
        |\{i + ar \pmod n \;|\; a \in \Z_n\}| = |\{ar \pmod n \;|\; a \in \Z_n\}| = \frac{n}{d}
    \]
    Portanto, fixando um elemento, determinamos $n/d$ outros. Como a pulseira tem $n$ elementos, podemos fazer $d$ escolhas entre as $k$ cores. Logo 
    $|N(r)| = k^{d}$. Substituindo no Lema,
    \[
        \#\{\text{pulseiras}\} = \frac{1}{n} \sum_{r = 0}^{n-1} k^{(r,n)} = 
        \frac{1}{n}\sum_{d|n}\varphi(d)k^{n/d}
    \]
    onde a última igualdade segue do fato que
    \[
        \sum_{i=1}^{n} k^{(r,n)} = \sum_{i=1}^{n} \sum_{d \mid n} \mathds{1}[d = (i,n)] k^{d} = 
        \sum_{d \mid n} k^{d} \cdot \bigg(\sum_{i=1}^{n} \mathds{1}[(i,n) = d]\bigg) = \sum_{d \mid n} k^d \varphi(n/d)
    \]
\end{proof}

\begin{proof}
    (Segunda prova mais fácil)
\end{proof}

\begin{problem}
    Dadas duas funções $f,g : \N^* \to \C$, definimos a convolução de Dirichlet $f * g : \N^* \to \C$ de $f$ e $g$ por 
    \begin{equation}
        \label{l3:prob2:a}
        f*g(n) := \sum_{d\mid n} f(d) g(n/d) = \sum_{d_1 d_2 = n} f(d_1)g(d_2)
    \end{equation}
\end{problem}
\begin{enumerate}[label=(\alph*)]
    \item Prove que, se $s \in \C$ e as séries $\sum_{n\geq 1} \frac{f(n)}{n^s}$ e $\sum_{n\geq 1}\frac{g(n)}{n^s}$ convergem
    absolutamente, então
    \[
        \sum_{n\geq 1} \frac{f(n)}{n^s} \cdot \sum_{n\geq 1}\frac{g(n)}{n^s} = \sum_{n\geq 1}\frac{(f*g)(n)}{n^s}
    \]

    \begin{proof}
        Como as séries convergem absolutamente, o produto converge absolutamente e podemos reordenar os índices de qualquer forma (contanto que apareçam todos).
        Portanto, 
        \[
            \sum_{n\geq 1} \frac{f(n)}{n^s} \cdot \sum_{m\geq 1}\frac{g(m)}{n^s} = \sum_{n,m \geq 1} \frac{f(n)g(m)}{(nm)^s}
            = \sum_{n\geq 1} \frac{1}{n^s} \cdot \bigg(\sum_{d_1 d_2 = n} f(d_1)g(d_2)\bigg) = \sum_{n\geq 1} \frac{(f*g)(n)}{n^s}.
        \]
    \end{proof}


    \item Prove que para quaisquer funções $f,g,h : \N^* \to \C$, temos $f*g = g*f$ e $f*(g*h) = (f*g)*h$ (comutatividade e associatividade), e 
    que a função $I : \N^* \to \C$ dada por
    \[
        I(n) = \begin{cases}
            1\quad \text{se } n = 1\\
            0\quad \text{se } n > 1
        \end{cases}
    \]
    é o elemento neutro da operação $*$.

    \begin{proof}
        A comutatividade segue diretamente da involução nos índices $d \to n/d$, segue que 
        \[
            f*g(n) = \sum_{d\mid n} f(d) g(n/d) = \sum_{d' \mid n} f(n/d')g(d') = g*f(n).
        \]

        Associatividade é vista melhor da segunda forma, 
        \[
        f*(g*h)(n) = \sum_{d_1x = n} f(d_1) (g*h)(x) = \sum_{d_1x = n} \sum_{d_2 d_3 = x} f(d_1) g(d_2) h(d_3) =  \sum_{d_1 d_2 d_3 = n} f(d_1) g(d_2) h(d_3)
        \]
        da mesma forma, 
        \[
            (f*g)*h(n) = \sum_{x d_3 = n} (f*g)(x)g(d_3) = \sum_{xd_3 = n} \sum_{d_1 d_2 = x} f(d_1) g(d_2) h(d_3) =  \sum_{d_1 d_2 d_3 = n} f(d_1) g(d_2) h(d_3)
        \]

        Por fim, para todo $n \in \N^*$,
        \[
            (I*f)(n) = \sum_{d \mid n} I(d)f(n/d) = f(n)
        \]
        logo, $I*f = f$.
    \end{proof}

    \item Prove que se $f$ e $g$ são multiplicativas, então $f*g$ é multiplicativa.
    \begin{proof}
        Sejam $n,m \in \N^*$ com $(n,m) = 1$, então $(f*g)(nm) = \sum_{dd' = nm}f(d)g(d')$. Escrevendo $d = (d,n)\cdot(d,m) = d_n d_m$
        e $d' = (d',n)(d',m) = d_n'd_m'$, como $dd'= nm$, devemos ter que $(d_n,d_m) = (d_n', d_m') = 1$,  $d_nd_n' = n$ e $d_md_m'=m$. Logo,
        \begin{align*}
            (f*g)(nm) &= \sum_{\substack{(d_n d_n' = n) \\ (d_m d_m' = m)}} f(d_nd_m)g(d_n'd_m') = \sum_{\substack{(d_n d_n' = n) \\ (d_m d_m' = m)}} f(d_n)f(d_m)g(d_n')g(d_m')\\
            &= \bigg(\sum_{(d_n d_n' = n)} f(d_n)g(d_n')\bigg) \bigg(\sum_{(d_m d_m' = m)} f(d_m)g(d_m')\bigg) = (f*g)(n) \cdot (f*g)(m) 
        \end{align*}
    \end{proof}

    \item Prove que, se $f : \N^* \to \C$ é tal que $f(1) \neq 0$, então existe uma única função $g : \N^* \to \C$ tal que $f*g = I$, 
    a qual é dada recursivamente por $g(1) = 1/f(1)$ e, para $n>1$
    \[
        g(n) = \frac{-1}{f(1)} \sum_{d \mid n, d\neq n} f(n/d)g(d).
    \]
    \begin{proof}
        A condição $(f*g)(1) = I(1) = 1$ determina únicamente que $g(1) = 1/f(1)$. Para $n > 1$, basta notar que deveríamos ter
        \[
            0 = I(n) = (f*g)(n) = \sum_{d \mid n} g(d) f(n/d) \iff f(1)g(n) = - \sum_{d\mid n, d\neq n} f(n/d)g(d)
        \]
        o que implica a fórmula.
    \end{proof}

    \item Prove que, se $f$ é multiplicativa então a função $f^{(-1)} = g$ definida no item anterior é multiplicativa.
    \begin{proof}
        Primeiramente, como $f$ é multiplicativa, $f(1) = 1 = g(1)$ (não pode ser $0$ senão não teria inversa). Vamos provar por indução em $n$ que 
        para todo $ab = n$ com $(a,b) = 1$, vale $g(n) = g(a)g(b)$ e portanto que $g$ é multiplicativa. O resultado segue para $n = 1$, suponha 
        que vale para todo $m < n$ e seja $n = ab$ com $(a,b) = 1$. Segue que
        \begin{align}
            g(ab) = - \sum_{\substack{d | ab\\d\neq ab}} g(d)f(ab/d) &= - \sum_{\substack{d_a | a, d_b \mid b\\ d_ad_b \neq ab}} g(d_a d_b)f\big(\frac{ab}{d_a d_b}\big)\\
            &= - \sum_{\substack{d_a | a, d_b \mid b\\ d_ad_b \neq ab}} g(d_a) g(d_b)f\big(\frac{a}{d_a}\big)f\big(\frac{b}{d_b}\big)\\
            \label{l3:eq:inversa_dirichlet}
            &= - \bigg[ \bigg(\sum_{d_a \mid a} g(d_a)f(a/d_a)\bigg)\bigg(\sum_{d_a \mid a} g(d_a)f(a/d_a)\bigg) - g(a)f(1)g(b)f(1)\bigg]
        \end{align}
        Agora, segue da fórmula do item anterior que
        \begin{align*}
            \sum_{d_a \mid a} g(d_a)f(a/d_a) = g(a)f(1) + \sum_{d_a \mid a, d_a \neq a} g(d_a)f(a/d_a) = g(a)f(1) - f(1)g(a) = 0,
        \end{align*}
        portanto [\ref{l3:eq:inversa_dirichlet}] se torna
        \[
            -[0 - g(a)g(b)f(1)^2] = g(a)g(b)
        \]
        pois $f(1) = 1$.
    \end{proof}
\end{enumerate}

\begin{problem}
    Prove que, se $\alpha > 2$, então
    \[
        \sum_{n=1}^{\infty} \frac{\varphi(n)}{n^\alpha} = \sum_{n=1}^{\infty} \frac{1}{n^{\alpha-1}} \bigg/ \sum_{n=1}^{\infty} \frac{1}{n^{\alpha}}.
    \]
\end{problem}
\begin{proof}
    Como $\alpha > 2$ e $1 \leq \varphi(n) \leq n$, todos os somatórios são absolutamente convergentes. Basta mostrar que
    \[
        \bigg(\sum_{n=1}^{\infty} \frac{\varphi(n)}{n^\alpha}\bigg)\bigg(\sum_{n=1}^{\infty} \frac{1}{n^{\alpha}}\bigg) = \sum_{n=1}^{\infty} \frac{1}{n^{\alpha - 1}}.
    \]
    Mas usando a equação [\ref{l3:prob2:a}] de (3.a) com $f(n) = \varphi(n)$ e $g(n) = 1$, temos 
    \begin{align*}
        \bigg(\sum_{n=1}^{\infty} \frac{\varphi(n)}{n^\alpha}\bigg)\bigg(\sum_{n=1}^{\infty} \frac{1}{n^{\alpha}}\bigg) &= \sum_{n=1}^{\infty} \frac{(f*g)(n)}{n^\alpha} =\\
        = \sum_{n=1}^{\infty} \frac{1}{n^\alpha} \sum_{d \mid n} \varphi(d) &= \sum_{n=1}^{\infty} \frac{n}{n^\alpha} = \sum_{n=1}^{\infty} \frac{1}{n^{\alpha -1 }}
    \end{align*}
    onde usamos o fato de que para todo $n \in \N^*$, vale que 
    \[
        \sum_{d \mid n} \varphi(d) = n.
    \]
    % Uma forma fácil de ver isso é checando
    % \[
    %     n = \sum_{i = 1}^{n} \sum_{d \mid n} \mathds{1}[(i,n) = d] = \sum_{d \mid n} \sum_{i=1}^{n} \mathds{1}[(i,n) = d]
    % \]
    % mas $(i,n) = d \iff (d\mid i)$ e $(i/d, n/d) = 1$, como há $\varphi(n/d)$ elementos em $\{1, \dots, n/d - 1\}$ primos 
    % com $n/d$, temos 
    % \[
    %     \sum_{i=1}^{n} \mathds{1}[(i,n) = d] = \varphi(n/d)
    % \]
    % e portanto
    % \[
    %     \sum_{d \mid n} \sum_{i=1}^{n} \mathds{1}[(i,n) = d] = \sum_{d \mid n} \varphi(n/d) = \sum_{d \mid n} \varphi(d).
    % \]
\end{proof}

\begin{problem}
    Mostre que o conjunto
    \[
        \bigg\{ \frac{\varphi(n)}{n} \;\bigg|\; n \in \N^* \bigg\}
    \]
    é denso em $[0,1]$.
\end{problem}

Precisaremos de um Lema fácil.
\begin{lemma}
    Seja $\sum_{n=1}^{\infty} a_n = \infty$ uma série divergente de termos positivos com 
    $a_n \to 0$, então o conjunto 
    \[
        M = \bigg\{ \sum_{n \in A} a_n \mid A \subset \N, |A| < \infty \bigg\}
    \]
    é denso em $(0,\infty)$ e portanto em $[0,\infty)$.
\end{lemma}
\begin{proof}
    Dados quaisquer $x > 0$ e $\eps > 0$, queremos encontrar um ponto de $M$ em $(x-\eps, x+\eps)$. Como $a_n \to 0$,
    existe $N$ tal que $\forall m \geq N$, $a_m < \min(\eps, x)$. Como $a_N < x$, $\sum_{n\geq N} a_n = \infty$ e os $a_n$ são positivos, 
    segue que existe um índice mínimo $j > N$ tal que
    \[
        x + \eps < \sum_{n = N}^{j+1} a_n
    \]
    e portanto, 
    \[
        x + \eps > \sum_{n = N}^{j} a_n = \bigg(\sum_{n = N}^{j+1} a_n\bigg) - a_{j+1} \geq (x+\eps) - \eps = x
    \]
    ou seja, se $A = \{n \;|\; N \leq n \leq j\}$, segue que $\sum_{n \in A} a_n \in (x-\eps, x+\eps)$.
\end{proof}

Agora podemos dar seguimento à solução do problema.
\begin{proof}
    (Do exercício) Escrevendo
    \[
       \frac{\varphi(n)}{n} = \prod_{p \mid n}\bigg(1 - \frac{1}{p}\bigg)
    \]
    com os $p$ primos. Vemos que o problema se resume a determinar se o conjunto 
    \[
        S = \bigg\{ \prod_{p \in A}\big(1 - \frac{1}{p}\big) \; \mid \; A \text{ conjunto finito de primos }\bigg\}
    \]
    é denso em $[0,1]$. Tomando $-\log$ (que é uma função contínua) em $S$, é equivalente também a verificar que
    \[
        -\log(S) = \bigg\{\sum_{p \in A} - \log(1 - \frac{1}{p}) \; \mid \; A \text{ conjunto finito de primos }\bigg\}
    \]
    é denso em $\R^+$. Mas, lembrando que $x - x^2/2\leq \log(1 + x) \leq x$ para $x > 1$, segue que
    \[
    \sum_{p_n \text{ primo}}-\log(1 - \frac{1}{p_n}) \geq \sum_{p_n \text{ primo}} \frac{1}{p_n} = \infty
    \]
    e que $0 < -\log(1 - \frac{1}{p_n}) < \frac{1}{p_n} + \frac{1}{2p_n^2} \to 0$ quando $n \to \infty$. Portanto, pelo Lema anterior,
    $-\log(S)$ é denso em $[0, \infty)$, logo $S$ é denso em $(0,1)$ e por consequência em $[0,1]$.
\end{proof}

\begin{problem}
    Seja $f : \N \to [0,\infty)$ uma função decrescente. Prove que a série
    \[
        \sum_{p \text{ primo}} f(p)
    \]
    converge se, e somente se, a série
    \[
        \sum_{n = 2} \frac{f(n)}{\log(n)}
    \]
    converge. Em particular, mostre que $\sum_{p \text{ primo}} 1/p = +\infty$, mas $\sum_{p \text{ primo}} 1/(p\log(p)) < \infty$ converge.
\end{problem}

Tem um leminha útil para esse problema.

\begin{lemma}
    Seja $f : [0,\infty) \to [0,\infty)$ uma função decrescente. Para toda constante $C > 0$, $\sum_{n\geq0}f(n) < \infty$ se, e somente se,
    $\sum_{n\geq0}f(Cn) < \infty$.
\end{lemma}


\begin{proof}
    Basta mostrar a ida, pois a recíproca é equivalente tomando $g(n) = f(Cn)$. Se $C \geq 1$, $f(n) \geq f(Cn)$ e
    \[
    \sum_{n\geq0}f(Cn) < \sum_{n\geq0}f(n) < \infty.
    \]  
    Se $0 < C < 1$, tome $M \in \N$ tal que $1/M \leq C$, vale que 
    \[
        \sum_{n = 0}^{\infty} f(Cn) \leq \sum_{n=0}^{\infty} f\bigg(\frac{n}{M}\bigg) = 
        \sum_{k = 0}^{\infty} \sum_{n = kM}^{(k+1)M - 1} f\bigg(\frac{n}{M}\bigg) \leq \sum_{k = 0}^{\infty} Mf(k) < \infty.
    \]    
\end{proof}


\begin{proof}
    (Do exercício) Pelo Teorema de Chebyshev, existem constantes $0 < A < B$ tal que, se $p_n$ é o $n$-ésimo primo, vale que 
    \[
        An\log n < p_n < B n\log n, 
    \]
    portanto, se $\sum_{p_n} f(p_n) < \infty$, segue que 
    \[
        \sum_{n \geq 1} f(Bn\log n) \leq \sum_{n \geq 1} f(p_n) < \infty.
    \]
    Pelo lema anterior, chamando $g(n) = f(Bn\log n)$, segue que para todo $C > 0$, $\sum_{n \geq 1} g(Cn) < \infty$.
    Em particular, tomando $C < \frac{A}{B}$, segue que para $M$ grande o suficiente, se $n \geq M$, então $An\log n > CBn\log(Cn)$
    e portanto,
    \[
        \sum_{n \geq M} f(A n\log n) \leq \sum_{n \geq M} f(CBn\log(n) + n\log(C/B)) \leq \sum_{n \geq 1} g(Cn) < \infty.
    \]
    Na verdade, trocando $A$ por qualquer constante positiva $D$, provamos que $\sum_{n \geq 1} f(Dn\log n) < \infty$. Portanto,
    \[
        \sum_{p_n} f(p_n) < \infty \quad \iff\quad \sum_{n \geq 1} f(n\log n) < \infty.
    \]
    Vamos mostrar que $\sum_{n \geq 1} f(n\log n) < \infty$ se, e somente se, $\sum_{n \geq 2} f(n)/\log n < \infty$. Chamando $w(n) = n\log(n)$,
    podemos escrever
    \[
        \sum_{n=2}^{\infty} \frac{f(n)}{\log(n)} = C + \sum_{n=3}^{\infty} \sum_{m = \floor{w(n)}}^{\floor{w(n+1)} - 1}\frac{f(m)}{\log(m)}
    \]
    para algum $C > 0$ que lida com os primeiros termos omitidos. Note que
    \[
        w(n+1) - w(n) = (n+1)\log(n+1) - n\log(n) = \Theta(\log(n))
    \]
    e \[
        \log(w(n)) = \log(n\log(n)) = \log(n) + \log\log(n) = \Theta(\log(n)).
    \]
    Segue que
    \[
        \sum_{n=3}^{\infty} \sum_{m = \floor{w(n)}}^{\floor{w(n+1)} - 1}\frac{f(m)}{\log(m)} \leq \sum_{n = 3}^{\infty} (w(n+1) - w(n))\frac{f(w(n))}{\log(w(n))} = \sum_{n = 3}^{\infty} \Theta(f(w(n))) < \infty
    \]
    portanto $\sum_{n \geq 1} f(w(n)) < \infty$ implica $\sum_{n \geq 2} f(n)/\log(n) < \infty$ e da mesma forma,
    \[
        \sum_{n=3}^{\infty} \sum_{m = \floor{w(n)}}^{\floor{w(n+1)} - 1}\frac{f(m)}{\log(m)} \geq \sum_{n = 3}^{\infty} (w(n+1) - w(n) - 1)\frac{f(w(n+1))}{\log(w(n+1))} = \sum_{n = 3}^{\infty} \Theta(f(w(n))) 
    \]
    e temos a volta, $\sum_{n \geq 2} f(n)/\log(n) < \infty \Rightarrow \sum_{n \geq 1} f(w(n)) < \infty$.

    Para a segunda parte da questão, basta verificar que 
    \[
        \sum_{n \geq 2} \frac{1}{n\log n} = \infty \quad \text{e} \quad \sum_{n \geq 2} \frac{1}{n(\log n)^2} < \infty
    \]
    mas segue do fato que 
    \[
        \sum_{n \geq 2}^{N} \frac{1}{n\log n} = \int_2^{N} \frac{dx}{x\log x}  + O(1) = \log \log N + O(1) \to \infty.
    \]
    Similarmente,
    \[
        \sum_{n \geq 2}^{\infty} \frac{1}{n(\log n)^2} = \int_2^{\infty} \frac{dx}{x(\log x)^2} + O(1) 
    \]
    fazendo a substituição $u = \log x$, $du = \frac{dx}{x}$, segue
    \[
        \int_2^{\infty} \frac{dx}{x(\log x)^2} + O(1) = \int_{\log(2)}^\infty \frac{du}{u^2} + O(1) < \infty
    \]
    como queriámos mostrar.
\end{proof}

\begin{problem}
    Um número de Sierpinski é um número natural ímpar $k$ tal que $2^n\cdot k + 1$ é composto para todo natural $n$. Prove que 
    78557 é um número de Sierpinski, e que existem infinitos números de Sierpinski a partir das congruências
    \begin{align*}
        78557\cdot2^0 + 1 \equiv 0 \pmod 3\\
        78557\cdot2^1 + 1 \equiv 0 \pmod 5\\
        78557\cdot2^7 + 1 \equiv 0 \pmod 7\\
        78557\cdot2^{11} + 1 \equiv 0 \pmod {13}\\
        78557\cdot2^3 + 1 \equiv 78557\cdot2^{39} + 1 \equiv 0 \pmod {73}\\
        78557\cdot2^{15} + 1 \equiv 0 \pmod {19}\\
        78557\cdot2^{27} + 1 \equiv 0 \pmod {37}.
    \end{align*}
\end{problem}

\begin{proof}
    Seja $k \in \N$, vamos mostrar que se $k$ é um ímpar suficientemente grande que satisfaz as mesmas recorrências que $78557$, então $k$ é de Sierpinski. 

    Suponha que $k\cdot 2^n + 1$ é primo e $k > 73$, segue que $k\cdot 2^n + 1 \not \in \{3,5,7,13,73,19,37\}$ e
    se $k\cdot2^n + 1 \equiv 0$ módulo qualquer um desses primos, $k\cdot 2^n + 1$ seria composto.  

    Como $2^2 \equiv 1 \pmod 3$, se $n \equiv 0 \pmod 2$, então 
    $k\cdot 2^n + 1 \equiv k\cdot 1 + 1 \equiv 0 \pmod 3$. Logo podemos supor que $n \equiv 1 \pmod 2$.

    Como $2^4 \equiv 1 \pmod 5$, então se $n \equiv 1 \pmod 4$, segue que $k\cdot 2^n + 1 \equiv k\cdot 2^1 + 1 \equiv 0 \pmod 5$. Podemos 
    supor que $n \equiv 3 \pmod 4$.

    De $2^3 \equiv 1 \pmod 7$, segue que se $n \equiv 1 \pmod 3$, então $k\cdot 2^{n} + 1 \equiv k\cdot 2 + 1 \equiv k\cdot 2^7 + 1 \equiv 0 \pmod 7$.
    Portanto, temos que $n \equiv \{0,2\} \pmod 3$. O que, junto com anterior, implica que $n \equiv \{3, 11\} \pmod {12}$.


    Como $2^{12} \equiv 1 \pmod {13}$, se $n \equiv 11 \pmod {12}$, então $k\cdot 2^n + 1 \equiv k\cdot2^{11} + 1 \equiv 0 \pmod {13}$, logo 
    podemos supor $n \equiv 3 \pmod {12}$.


    De $78557\cdot2^3 + 1 \equiv 78557\cdot2^{39} + 1\pmod {73}$ e $(78557,73) = 1$, segue que $2^{36} \equiv 1 \pmod {73}$ e, 
    portanto, se $n \equiv 3 \pmod {36}$, então $k\cdot 2^n + 1 \equiv k\cdot 2^{3} + 1 \equiv 0 \pmod {73}$. Ou seja, podemos supor $n \equiv \{15, 27\} \pmod {36}$.

    Como $2^{18} \equiv 1 \pmod {19}$, então se $n \equiv {15} \pmod 18$, temos $k\cdot 2^{n} + 1 \equiv k\cdot 2^{15} + 1 \equiv 0 \pmod {19}$.
    Ou seja, $n \not \equiv 15 \pmod {36}$ e devemos ter que $n \equiv 27 \pmod {36}$.

    De $2^{36} \equiv 1 \pmod {37}$, se $n \equiv 27 \pmod {36}$, então $k\cdot 2^{n} + 1 \equiv k\cdot 2^{27} + 1 \equiv 0 \pmod {37}$. Mas isso é absurdo,
    já tinhamos restringido $n$ para $27 \pmod {36}$.

    Tomando $k \equiv 1 \pmod 2$, forçamos $k$ a ser ímpar, pelo TCM, existem infinitas soluções para $k > 73$, todas elas de Sierpinski (inclusive 78557). 
\end{proof}



\end{document}
